\usepackage[utf8]{inputenc}
% \usepackage[T1]{fontenc}
% \usepackage{lmodern}

\usepackage{latexsym}
\usepackage{amssymb,amsmath}
\usepackage{svg}
\usepackage{subfig}
% \usepackage[pdftex]{graphicx}
% \graphicspath{{graphics/}}
\usepackage[english, science, titlepage]{Preamble/ku-frontpage}
\usepackage{float}
\usepackage{caption}
\usepackage[toc,page]{appendix}
\usepackage{fancyvrb}
\usepackage{url}
\usepackage{tabularx}
\usepackage{booktabs}
\setlength\parindent{0pt}
\usepackage{titlesec}
% \usepackage{emoji}
\usepackage{semantic}
\usepackage{stmaryrd}
\usepackage{hyperref}
\usepackage[noabbrev,capitalise]{cleveref}
\usepackage{multirow}
\usepackage{longtable}

\usepackage[backend=biber,dateabbrev=false]{biblatex}
\usepackage{csquotes} % package to biblatex
\addbibresource{references.bib}

\titleformat{\chapter}[display]
{\normalfont\huge\bfseries}{\chaptertitlename\ \thechapter}{20pt}{\Huge}

% this alters "before" spacing (the second length argument) to 0
\titlespacing*{\chapter}{0pt}{-60pt}{10pt}
\raggedbottom
\makeatletter
\providecommand\phantomsection{}% for hyperref

\newcommand\listofillustrations{%
    \chapter*{List of Illustrations}%
    \phantomsection
    \addcontentsline{toc}{chapter}{List of Illustrations}%
    \section*{Figures}%
    \phantomsection
    \addcontentsline{toc}{section}{\protect\numberline{}Figures}%
    \@starttoc{lof}%
    \bigskip
    \section*{Tables}%
    \phantomsection
    \addcontentsline{toc}{section}{\protect\numberline{}Tables}%
    \@starttoc{lot}}

\makeatother
%   Reduce the margin of the summary:
\def\changemargin#1#2{\list{}{\rightmargin#2\leftmargin#1}\item[]}
\let\endchangemargin=\endlist

%   Generate the environment for the abstract:
\newcommand\summaryname{Abstract}
\newenvironment{Abstract}%
    {\small\begin{center}%
        \bfseries{\summaryname} \end{center}}

\usepackage{syntax}
\setlength{\grammarparsep}{8pt plus 1pt minus 1pt} % increase separation between rules
\setlength{\grammarindent}{7em} % increase separation between LHS/RHS

\usepackage{listings}
\usepackage{xcolor}

\definecolor{codegreen}{rgb}{0,0.6,0}
\definecolor{codegray}{rgb}{0.5,0.5,0.5}
\definecolor{codepurple}{rgb}{0.58,0,0.82}
\definecolor{backcolour}{rgb}{0.95,0.95,0.92}

\lstdefinestyle{mystyle}{
    backgroundcolor=\color{backcolour},
    commentstyle=\color{codegreen},
    keywordstyle=\color{magenta},
    numberstyle=\tiny\color{codegray},
    stringstyle=\color{codepurple},
    basicstyle=\ttfamily\footnotesize,
    breakatwhitespace=false,
    breaklines=true,
    captionpos=b,
    keepspaces=true,
    numbers=left,
    numbersep=5pt,
    showspaces=false,
    showstringspaces=false,
    showtabs=false,
    tabsize=2
}

\lstset{style=mystyle}

% Left-right absolute value
\newcommand{\lra}[1]{\langle #1 \rangle}

\newcommand{\judge}[2]{\lra{#1} \Downarrow #2}

\newcommand{\hoare}[3]{\{#1\}#2\{#3\}}

\newcommand{\hoarePQ}[1]{\{P\}#1\{Q\}}