In \autoref we defined the big-step semantics of WHILE.
In both why3 and Isabelle it is possible to create inductive predicates.
An inductive predicate is simply a predicate with a set of proof constructors defined inductively.
This is a great tool to make a close to one to one mapping from the pen and paper version and a formalized
version.
We define the semantics of arithmetic expressions as follows:
\begin{lstlisting}[caption={semantics of arithmetic expressions},label={lst:why3aexpr},language=sml]
type e_behaviour 'a = Enormal 'a | Eabnormal

inductive eval_aexpr (aexpr 'a) (store 'a) (e_behaviour int) =
  | EACst : forall n, s : store 'a.
             eval_aexpr (Acst n) s (Enormal n)
  | EAVar : forall x n, s : store 'a.
             s[x] = Some n ->
             eval_aexpr (Avar x) s (Enormal n)
  | EAVar_err : forall x, s : store 'a.
             s[x] = None ->
             eval_aexpr (Avar x) s Eabnormal
  | EASub : forall a1 a2 n1 n2, s : store 'a, op.
             eval_aexpr a1 s (Enormal n1) ->
             eval_aexpr a2 s (Enormal n2) ->
             op = Sub ->
             eval_aexpr (ABin a1 op a2) s (Enormal (n1 - n2))
  | EAMul : forall a1 a2 n1 n2, s : store 'a, op.
             eval_aexpr a1 s (Enormal n1) ->
             eval_aexpr a2 s (Enormal n2) ->
             op = Mul ->
             eval_aexpr (ABin a1 op a2) s (Enormal (n1 * n2))
  | EAOp_err1 : forall a1 a2, s : store 'a, op.
             eval_aexpr a1 s Eabnormal ->
             eval_aexpr (ABin a1 op a2) s Eabnormal
  | EAOp_err2 : forall a1 a2, s : store 'a, op.
             eval_aexpr a2 s Eabnormal ->
             eval_aexpr (ABin a1 op a2) s Eabnormal
\end{lstlisting}

From \autoref{lst:why3aexpr} it should be straight forward to see that each of the predicates, corresponds to
each of the inference rules in \autoref{}.
Each subterm before \texttt{->} is the premises of the specific inference rule and the last term is the conclusion.
There is however a slight difference.
Initially we split the EAOp rule into an EAMul and EASub, since we though this would be a necessity,
and more closely resembles how semantics usually are represented in text books and as such should be considered independently.
However we found that this indeed is not necessary because of the rather trivial and mechanized reasoning about such binary operators and in fact reduces complexity of the automated proofs as is apparent in \autoref{fig:aexpr_props}.
The figure shows the 3 goals for arithmetic expressions. For a description of each property see \autoref{TODO}.
In both versions of the semantics the goals are Valid, however we see a big difference in the steps it takes to validate it.
For the initial version the steps for Goal 1 and Goal 3 takes roughly 2.5 as many steps as the current version. For Goal 2 the initial version requires 4 times as many steps as the current.

\begin{figure}
\begin{lstlisting}
INITIAL VERSION (version where we specifically distingush the two cases)
File semantics.mlw:
Goal eval_aexpr_total_fun'vc.
Prover result is: Valid (0.19s, 4170 steps).

File semantics.mlw:
Goal eval_aexpr_fun'vc.
Prover result is: Valid (0.58s, 14543 steps).

File semantics.mlw:
Goal eval_aexpr_total'vc.
Prover result is: Valid (0.03s, 428 steps).

NEW VERSION (Where we convert aop to a why3 binary operator).
File semantics.mlw:
Goal eval_aexpr_total_fun'vc.
Prover result is: Valid (0.09s, 1577 steps).

File semantics.mlw:
Goal eval_aexpr_fun'vc.
Prover result is: Valid (0.20s, 3449 steps).

File semantics.mlw:
Goal eval_aexpr_total'vc.
Prover result is: Valid (0.01s, 105 steps).

\end{lstlisting}
\caption{Steps to prove properties for arithmetic expressions}
\label{fig:aexpr_props}
\end{figure}

The new and improved semantic representation makes uses of a function \texttt{eval\_op :: axpr -> (int -> int -> int)} and used it in the conclusion as \texttt{eval\_aexpr (ABin a1 op a2) s (Enormal ((eval\_op op) a1 a2)}.
Hence we get a direct correspondence to the semantics presented in \autoref{fig:aexprsemantics}, by removing EAMul and EASub and replacing it with

\begin{lstlisting}
let function eval_op (op : aop) : (int -> int -> int) =
    match op with
    | Mul -> (*) | Sub -> (-)
    end
...

  | EABin : forall a1 a2 n1 n2, s : store 'a, op.
             eval_aexpr a1 s (Enormal n1) ->
             eval_aexpr a2 s (Enormal n2) ->
             eval_aexpr (ABin a1 op a2) s (Enormal ((eval_op op) n1 n2))
\end{lstlisting}

Notice furthermore we can define the behaviour of an operation as a Discriminated union of either a polymorphic \texttt{Enormal} behaviour or an \texttt{Eabnormal error},
we do so to reuse the same types for arithmetic and boolean expressions along with statements.
In the current implementation the only way to not result in normal behaviour is when we have an unbound variable.

If we in the future were to add additional errors we let Eabnormal hold an ADT containing different errors.
We tried looking into this matter, by adding division as an operator and adding additional semantics,
with possibility of resulting in an \texttt{Ebnormal Ediv0} error. We however did not find a way to prove this,
as the SMT solvers would time out on the properties. It is hard to tell exactly why as, adding addition as an operator we get only a fairly small increase in number of steps.
TODO what are the numbers.
But the additional error cases might make the actual formula discharged be grow exponetially. Considering the structure of the inductive predicate, we cannot simply let division be part of
Binop, thus we would need seperate rules for the different operators.

% The equivalent predicate in Isabelle is nearly identical except for a few syntactical differences, for
% instance we dont have to explicitly instantiate each variable as they implicitly instantiated to be fixed but arbitrary.

% \begin{minted}
% datatype 'ty behaviour = Enormal 'ty | Eunbound

% fun eval_binop :: "aop ⇒ (int ⇒ int ⇒ int)" where
%   "eval_binop Mul = (*)"
% | "eval_binop Sub = (-)"

% inductive eval_aexpr :: "int aexpr ⇒ int state ⇒ int behaviour ⇒ bool" where
%     EACst     : "eval_aexpr (Cst n) s (Enormal n)"
%   | EAVar     : "s x = Some n ⟹
%                  eval_aexpr (Var x) s (Enormal n)"
%   | EAVar_err : "s x = None ⟹
%                  eval_aexpr (Var x) s Eunbound"
%   | EABin     : "eval_aexpr a1 s (Enormal n1) ⟹
%                  eval_aexpr a2 s (Enormal n2) ⟹
%                  eval_aexpr (BinOp a1 op a2) s (Enormal ((eval_binop op) n1 n2))"
%   | EABin_err1 : "eval_aexpr a1 s Eunbound ⟹
%                   eval_aexpr (BinOp a1 op a2) s Eunbound"
%   | EABin_err2 : "eval_aexpr a1 s (Enormal n) ⟹
%                   eval_aexpr a2 s Eunbound ⟹
%                   eval_aexpr (BinOp a1 op a2) s Eunbound"
% \end{minted}

% The interesting thing here, is that we use the other proposed semantics, the one which follows the rules directly.
% The reason for this distiction is because while we can prove certain properties in why3 with one semantic
% and certain properties in Isabelle. For instance, if we take a look at determinism for arithmetics expressions,
% which we introduced in \autoref{}
