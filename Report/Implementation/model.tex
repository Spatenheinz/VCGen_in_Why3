To simulate the store, we had to implement a seperate module \texttt{ImpMap}, presented in \autoref{lst:why3state},
since we ran into a number of roadblocks using the predefined data-structures in the standard library.
These will be explained further in \ref{sec:results}.
\\~\\
We define the store through a record called ``state''.
This record contains two different fields.
Firstly we have \texttt{lst} which is a linked list of a key value pair,
whilst the second field \texttt{model} is a mapping of the same type as the state used for defining the semantics.
Notice here that the model is marked with the ghost keyword. This means that the model can only be used in a logical context
and therefore cannot manipulate any computations.
We use the model for reasoning about the store through the verification conditions.

The module implements three functionalities, \texttt{empty}, \texttt{add} and \texttt{find}.

Instantiation of a state by \texttt{empty} should ensure that the image of the store is \{\texttt{None}\}.

You can add key value pairs to a state using \texttt{add}, which will add the pair to the existing mappings.

Lastly, \texttt{find} tries to find the key in the list and if it does not exist we throw an error.

This use of the model enables us to easily propagate the error throughout the evaluation.
The full implementation can be seen in \autoref{appendix somewhere}.

% \begin{lstlisting}[caption={Model for a store},label={lst:why3state},language=sml]
% type model_t = M.map int (option int)

% predicate match_model (k: int) (v: int) (m : model_t) =
% match M.get m k with
% | None -> false
% | Some v' -> v = v'
% end

% function helper (m : model_t) (pair: (int,int)) : bool =
% let (k,v) = pair in match_model k v m

% type state = { mutable lst : list (int, int);
%                ghost mutable model : model_t
%              }
%       (* invariant { for_all (helper model) lst } *)
%       (* by { lst = Nil ; model = (fun (_ : int) -> None) } *)

% exception Unbound

% function domain (s : state) : M.map int (option int) = s.model

% let function empty () : state
% ensures {forall k. M.get result.model k = None }
% = { lst = Nil ; model = (fun (_ : int) -> None) }

% let add (k: int) (v: int) (s : state) : ()
% writes { s.lst }
% writes { s.model }
% ensures { s.model = M.((old s.model)[k <- Some v]) }
% ensures { hd s.lst = Some (k,v) /\ match_model k v s.model }
% = s.lst <- (Cons (k,v) s.lst);
%   s.model <- M.(s.model[k <- Some v])

% let rec find (k: int) (s : state) : int
%   variant { s.lst }
%   ensures { match_model k result s.model}
%   raises { Unbound -> M.get s.model k = None}
% = match s.lst with
%   | Cons (k', n) s' -> if andb (k <= k') (k' <= k) then n
%      else find k { lst = s'; model = s.model}
%   | Nil -> raise Unbound
%   end
% \end{lstlisting}
