\section{Implementation}\label{sec:implementation}
In this section we describe how we implemented the presented syntactical and semantical system
in Why3. This is presented for each syntactical component: arithmetic expressions, boolean expression, statements, and formulas. In \ref{sec:evaluator} we describe how we implemented
an evaluator for the language, and in \ref{sec:vcg} the implementation of verification
condition generation is presented.

For each component we define a set of properties which must hold for a correct implementation.
We sometimes accompany the Why3 implementations with proofs conducted in Isabelle, to reason
about parts that were tricky in Why3.

The entire implemetation can be seen in \cref{code}.

\subsection{Syntax}
Firstly our store $\sigma$ is defined as a mapping from a polymorphic type to an Option Int.
We define the syntax by simple ADTs, which closely resembles the syntax described earlier.
\\~\\
\noindent
\begin{minipage}{0.5\textwidth}
\begin{lstlisting}
type store 'a = 'a -> option int

type aop = Mul | Sub

type aexpr 'a =
     | Acst int
     | Avar 'a
     | ABin (aexpr 'a) aop (aexpr 'a)

type bexpr 'a =
     | Btrue
     | Bfalse
     | Bleq (aexpr 'a) (aexpr 'a)
     | Band (bexpr 'a) (bexpr 'a)
     | Bnot (bexpr 'a)
\end{lstlisting}
\end{minipage}
\hspace{0.01\textwidth}
\begin{minipage}{0.5\textwidth}
\begin{lstlisting}
type_synonym 'a state = "'a \<Rightarrow> int option"

datatype aop = Mul | Sub

datatype 'a aexpr =
    Cst int
  | Var 'a
  | BinOp "'a aexpr" aop "'a aexpr"
\end{lstlisting}
\end{minipage}

\subsection{Arithmetic Expressions}
In the following we will describe how the semantics for arithmetic expressions are implemented,
and which properties we want to hold to ensure correctness of the implementation.

\subsubsection{Semantics}
In \ref{sec:semantics} we defined the big-step semantics of WHILE.
To convert these semantics into Why3 we use inductive predicates.
An inductive predicate in Why3 is simply a predicate with a set of proof constructors defined inductively.
This is a great tool to make an almost one to one mapping from the pen and paper version and a formalized version.
We define the semantics of arithmetic expressions in Why3 as follows:

\begin{lstlisting}[caption={Semantics of arithmetic expressions},label={lst:why3aexpr},language=sml]
  type e_behaviour 'a = Enormal 'a | Eabnormal error

  let function eval_op (op : aop) : (int -> int -> int) =
      match op with
      | Mul -> (*) | Sub -> (-) | Add -> (+)
      end

  inductive eval_aexpr (aexpr 'a) (store 'a) (e_behaviour int) =
    | EACst : forall n, s : store 'a.
               eval_aexpr (Acst n) s (Enormal n)
    | EAVar : forall x n, s : store 'a.
               s[x] = Some n ->
               eval_aexpr (Avar x) s (Enormal n)
    | EAVar_err : forall x, s : store 'a.
               s[x] = None ->
               eval_aexpr (Avar x) s (Eabnormal Eunbound)
    | EABin : forall a1 a2 n1 n2, s : store 'a, op.
               eval_aexpr a1 s (Enormal n1) ->
               eval_aexpr a2 s (Enormal n2) ->
               eval_aexpr (Abin a1 op a2) s
                          (Enormal ((eval_op op) n1 n2))
    | EABin_err1 : forall a1 a2, s : store 'a, op e.
               eval_aexpr a1 s (Eabnormal e) ->
               eval_aexpr (Abin a1 op a2) s (Eabnormal e)
    | EABin_err2 : forall a1 a2 n1, s : store 'a, op e.
               eval_aexpr a1 s (Enormal n1) ->
               eval_aexpr a2 s (Eabnormal e) ->
               eval_aexpr (Abin a1 op a2) s (Eabnormal e)
\end{lstlisting}

From \autoref{lst:why3aexpr} it can be seen that each of the predicates, corresponds to
each of the inference rules in \autoref{fig:aexprsemantics}.
Each subterm before the \texttt{->} symbol is the premise of the specific inference rule, and the last term is the conclusion.
The name before the \texttt{:} is the name of that specific
rule, and these correspond to those of the semantic inference rules.
\\~\\
At first we had semantic inference rules for each of the binary operators, as we assumed this
would be necessary to prove correctness, and as it
more closely resembles how semantics usually are represented in text books, indicating that operators as such should be considered independently.
However we found that this indeed is not necessary to conduct the proofs, because of the rather
trivial and mechanized reasoning about such binary operators.
In fact it reduces complexity of the automated proofs as is apparent in \autoref{fig:aexprprops}.
The figure shows 3 goals each corresponding to a property for arithmetic expressions, and how
many steps it takes to validate each of them, first for the version using different rules for
each operator, and afterwards for the version that combines them into one rule.
It should be noted that for these tests we only have two operators, namely subtraction and
multiplication, as we did not add addition until later.
For a description of each property see \ref{sec:aprops}.

In both versions of the semantics the goals are valid, however we see a big difference in the steps it takes to validate it.
For the initial version, where each operator has its own inference rules, the steps for Goal 1 and Goal 3 takes roughly 2.5 as many steps as the current version.
For Goal 2 the initial version requires 4 times as many steps as the current.

\begin{figure}
\begin{lstlisting}
INITIAL VERSION (version where we specifically distingush between operators)
File semantics.mlw:
Goal eval_aexpr_total_fun'vc.
Prover result is: Valid (0.19s, 4170 steps).

File semantics.mlw:
Goal eval_aexpr_fun'vc.
Prover result is: Valid (0.58s, 14543 steps).

File semantics.mlw:
Goal eval_aexpr_total'vc.
Prover result is: Valid (0.03s, 428 steps).

NEW VERSION (Where we convert aop to a why3 binary operator).
File semantics.mlw:
Goal eval_aexpr_total_fun'vc.
Prover result is: Valid (0.09s, 1577 steps).

File semantics.mlw:
Goal eval_aexpr_fun'vc.
Prover result is: Valid (0.20s, 3449 steps).

File semantics.mlw:
Goal eval_aexpr_total'vc.
Prover result is: Valid (0.01s, 105 steps).

\end{lstlisting}
\caption{Steps to prove properties for arithmetic expressions}
\label{fig:aexprprops}
\end{figure}

The new and improved semantic representation makes use of a function
\[
  \texttt{eval\_op :: axpr -> (int -> int -> int)}
\]
to ensure the correct conclusion of expressions with binary operators, using the rule
\[
  \texttt{eval\_aexpr (ABin a1 op a2) s (Enormal ((eval\_op op) a1 a2)}
\]
Hence we get a direct correspondence to the semantics presented in
\autoref{fig:aexprsemantics}.

% \begin{lstlisting}
% let function eval_op (op : aop) : (int -> int -> int) =
%     match op with
%     | Mul -> (*) | Sub -> (-)
%     end
% ...

%   | EABin : forall a1 a2 n1 n2, s : store 'a, op.
%              eval_aexpr a1 s (Enormal n1) ->
%              eval_aexpr a2 s (Enormal n2) ->
%              eval_aexpr (ABin a1 op a2) s (Enormal ((eval_op op) n1 n2))
% \end{lstlisting}

Notice furthermore we can define the behaviour of an operation as a discriminated union of
either a polymorphic \texttt{Enormal 'a} behaviour or an \texttt{Eabnormal error}.
We do so to reuse the same types for arithmetic and boolean expressions along with statements.

In the current implementation the only way to result in abnormal behaviour is when we have an unbound variable.
If we in the future were to add additional errors, we can extend the \texttt{error} type with more
instances of errors.
We tried looking into this matter, by adding division as an operator and adding additional semantics,
with possibility of resulting in an \texttt{Ebnormal Ediv0} error. We however did not find a way to prove this,
as the SMT solvers would time out on the properties. It is hard to tell exactly why as, but it
could be because the complexity is increased so much that the proof times out.
When adding addition as an operator we get a small increase in number of steps, but it could be
that a more complex operator, ie. division, adds too much complexity to the proofs.

Hence it could be that the additional error cases make the discharged formula grow
exponetially.
Considering the structure of the inductive predicate, we cannot simply include division in the
rules for binary operators, thus to include division we need new inference rules.
This would also make the proof grow, as there would be many more rules to consider.

% The equivalent predicate in Isabelle is nearly identical except for a few syntactical differences, for
% instance we dont have to explicitly instantiate each variable as they implicitly instantiated to be fixed but arbitrary.

% \begin{minted}
% datatype 'ty behaviour = Enormal 'ty | Eunbound

% fun eval_binop :: "aop ⇒ (int ⇒ int ⇒ int)" where
%   "eval_binop Mul = (*)"
% | "eval_binop Sub = (-)"

% inductive eval_aexpr :: "int aexpr ⇒ int state ⇒ int behaviour ⇒ bool" where
%     EACst     : "eval_aexpr (Cst n) s (Enormal n)"
%   | EAVar     : "s x = Some n ⟹
%                  eval_aexpr (Var x) s (Enormal n)"
%   | EAVar_err : "s x = None ⟹
%                  eval_aexpr (Var x) s Eunbound"
%   | EABin     : "eval_aexpr a1 s (Enormal n1) ⟹
%                  eval_aexpr a2 s (Enormal n2) ⟹
%                  eval_aexpr (BinOp a1 op a2) s (Enormal ((eval_binop op) n1 n2))"
%   | EABin_err1 : "eval_aexpr a1 s Eunbound ⟹
%                   eval_aexpr (BinOp a1 op a2) s Eunbound"
%   | EABin_err2 : "eval_aexpr a1 s (Enormal n) ⟹
%                   eval_aexpr a2 s Eunbound ⟹
%                   eval_aexpr (BinOp a1 op a2) s Eunbound"
% \end{minted}

% The interesting thing here, is that we use the other proposed semantics, the one which follows the rules directly.
% The reason for this distiction is because while we can prove certain properties in why3 with one semantic
% and certain properties in Isabelle. For instance, if we take a look at determinism for arithmetics expressions,
% which we introduced in \autoref{}


\subsubsection{Properties}\label{sec:aprops}
The properties we want to hold for arithmetic expressions are straightforward.
We want our program to be deterministic which means each part of the semantics must be deterministic, i.e. the semantic relation must be functional.
Furthermore we want semantics to be total, such that we never encounter undefined behaviour.
We already presented a formulation for determinism of arithmetic expressions in \autoref{lst:why3fun}.
This lemma cannot be proved directly.

\subsection{Boolean Expressions}
Now we will describe the implementation of semantics for boolean expressions, together with
the properties we want to ensure holds.

\subsubsection{Semantics}
The formulation of boolean expressions follows the same pattern as arithmetic expressions.
We define an inductive predicate stating the inference rules in \autoref{fig:bexprsemantics}.
As this follows the same structure as for arithmetic expressions we exclude the code here,
for reference see \autoref{appendix???}.

\subsubsection{Properties}
Likewise the properties we use for reasoning about correctness of our boolean semantics resemble the properties for arithmetic expressions. Thus we have the following lemmas:

\begin{enumerate}
  \item \texttt{eval\_bexpr\_fun}
  \item \texttt{eval\_bexpr\_total}
  \item \texttt{eval\_bexpr\_total\_fun}
\end{enumerate}

In the formulation of the boolean semantics, we initially encountered some problems in proving the properties.
In the first iteration of the inductive predicate we used the built in operator \&\&, which works for boolean values.
We then became aware of the function \texttt{andb}, which is defined in the standard library in module bool.Bool.
\texttt{andb} is short-circuiting and presumably the \&\& operator is as well.

From the properties it also seem the two functions have equivalent semantics.
Interestingly they do not require the same amount of steps in
the proofs. \autoref{tab:stepsbexpr} shows the result of proving the properties with different combinations of use of the two operators.
The columns define the operator used in the lemma \texttt{eval\_bexpr\_total\_fun},
and the rows define which operator is used in the inductive predicate.
Each line in a cell corresponds to the number of steps required for Alt-Ergo to prove the lemma, stated in the order from above.

From the figure, we can see that using \&\& makes the two sub-lemmas a little simpler to prove, however
for the total function lemma, it requires significantly more steps.
On the other hand \texttt{andb} requires a bit more steps for the two sub-lemmas but is more than 4 times more efficient for the total function lemma.
Most bizzarely is it that using \texttt{andb} in \texttt{eval\_bexpr} and using \&\& in \texttt{eval\_bexpr\_total\_fun} cannot even be shown, despite the fact that their semantics evidently should be the same.
As a result of these considerations we ended up using \texttt{andb} in our implementation.
\begin{table}
  \centering
  \begin{tabular}{c || c | c}
     predicate \textbackslash lemma & \&\& & andb \\
    \hline
    \hline
    \&\& & \begin{array}{r} 12008 \\ 1058 \\ 51046 \end{array} & \begin{array}{r} 56390 \\1058 \\ 51046 \end{array} \\
    \hline
    andb & \begin{array}{r} timeout \\ 1144 \\ 12146 \end{array} & \begin{array}{r} 14508 \\ 1144 \\ 12146 \end{array} \\
  \end{tabular}
  \caption{Table of the number of steps taken to prove the three properties.}
  \label{tab:stepsbexpr}
\end{table}

% && vs andb
% band


\subsection{Statements}
Finally we present our implementation of semantics of statements.
\subsubsection{Semantics}
Once again, we define the semantics by an inductive predicate. The semantics are generally not that complex,
however there are many more cases to account for when dealing with statements.
An example of this is the inference rules for while-statements.
We must consider 4 different cases:

\begin{enumerate}
  \item The boolean condition is true, and thus the body evaluates to some new state, which is used for the next iteration of the while loop.
  \item The boolean condition is false, and thus the loop ends in the same state.
  \item The boolean condition results in abnormal behaviour and thus the entire statement should result in abnormal behaviour.
  \item The body results in abnormal behaviour and likewise the entire loop results in abnormal behaviour.
\end{enumerate}

However for assertions, we only consider a single case, since, as mentioned when presenting the inference rules in \ref{sec:stmtsemantics},
we do not consider assertions in the operational semantics, but rather include them for strengthening the
verification condition of a program.

\subsubsection{Properties}
The properties for asserting the correctness of the implementation of the semantics is built
upon the lemmas regarding totality and determinism of boolean and arithmetic expressions.
Hence we consider the lemma
\texttt{eval_stmt_deterministic}. Unfortunately, our trick from the previous grammar constructs of utilising a recursive lemma does not render useful for statements. Running the recursive lemma for 300 seconds in Alt-Ergo, does not provide a proof.
There might be many reasons for this.

First and foremost we cannot include while in the recursive lemma. The reason for this, is that we have to
include the variant for a recursive lemma to even discharge the proof. I.e. if we cannot prove termination of a lemma it cannot hold true in the system if why3. Since this would only entail partial correctness.
It should be clear from the ESWhileT case of the inference rules, that we do not reduce the structure of s and hence s does not respect the well founded order.
Had we included a variant in WHILE we might have been able to express the termination.

We then considered the proof without including the While in the language. Also in this case we could not find a proof. We then tried to remove If. Yet again, the SMT solvers timed out. Finally we found that removing Sequences, would allow us to prove the determinism.

specifically, when excluding Seq, If, While, it took Alt-Ergo 5956 steps to prove the lemma.
While only excluding Seq and While the prover to 13965 steps. Again this shows how 3 additional inference rules
can increase the number of step by a significant amount.

One perculiarity we further found was that one can compare mappings by = without importing the module map.MapExt (in the standard library), which specifically defines extensionality of mappings.
But again this might be overlapping instances of the build in functionality and the standard library.
However this, along with the conumdrum of \&\& and \texttt{andb}, begs the question as to how well the
standard library is structured.



\subsection{Assertion Language (Formulas)}\label{sec:iformulas}
In this section we will describe the formalization of formulas, ie. the assertion language of WHILE.

\paragraph{Formalization.}
As mentioned in \ref{sec:assert}, we cannot formalize the assertion language in a similar manner to
the other language constructs because of quantifers.
We instead decided to use a predicate to describe the semantics.
As formulas are logic constructs the choice naturally was to use a predicate over a function,
since predicates are part of why3's logical languages and therefore is useful in reasoning about logics.

We want to formalize that the evaluation of a formula depends on the evaluation of subterms
of that formula. This will eventually use the evaluation of boolean expressions, and thus
the result will be whether the formula holds or not.
This is formalized in the predicate \texttt{eval\_closed\_formula}, shown in \autoref{lst:why3formula}.
\\~\\
The semantics for conjuction, implication and negation are rather simple and directly follows the corresponding
semantics of why3's logic.
Universal quantification uses the \texttt{forall} defined in why3, and updates the state accordingly.

For the term expression we use the semantics defined earlier for binary expressions.
We state that for a term to be true, the term should evaluate to true under the judgement of binary expressions.
If it evaluates to true, the term is trivially true.
If it does not evaluate to true, there are two potential evaluations.
Either it can evaluate to false or it results in $\bot$.
This poses a problem in formulas such as $\neg (x \leq 10)$ where $x$ is unbound.
The inner expression $x \leq 10$ will be a false term, as $x$ is unbound, and by negation the entire formula is true.
This should not be possible as formulas with unbound variables should not make sense.
Hence the semantics only works for closed formulas, and we therefore need a wrapper function, to ensure that we only consider closed formulas.

The wrapper \texttt{eval\_formula} will first check if any free variables exists, and
if this is the case the formula evaluates to false.
In case we have a closed formula, we evaluate the formula using the defined predicate.
\\~\\
To determine whether any free variables exists in the formula, we define a predicate
\texttt{is\_closed} which takes a formula and a store and recursively traverse the formula to ensure all occurring variables are bound in the given store.
In case of universal quantification, we bind the quantified variable to 0,
so it is present in the store. The value in this case is irrelevant, as we just want to determine whether the formula is closed or not, and not what it evaluates to.
The \texttt{eval\_formula} and \texttt{eval\_closed\_formula} predicates are shown in
\autoref{lst:why3formula}. For the rest of the implementation, see Appendix \ref{codeformulas}.

\begin{lstlisting}[caption={Predicate defining the semantics of formulas},label={lst:why3formula},language=sml]
predicate eval_closed_formula (f: formula 'a) (st: store 'a)  =
  match f with
  | Fterm b -> eval_bexpr b st (Enormal true)
  | Fand f1 f2 -> eval_closed_formula f1 st /\ eval_closed_formula f2 st
  | Fnot f -> not (eval_closed_formula f st)
  | Fimp f1 f2 -> eval_closed_formula f1 st -> eval_closed_formula f2 st
  | Fall y f -> forall n. eval_closed_formula f st[y <- n]
  end

predicate eval_formula (f : formula 'a) (st : store 'a) =
  if is_closed_formula f st then eval_closed_formula f st
  else false
\end{lstlisting}

\paragraph{Blocking and non-blocking.}
We mentioned briefly in \ref{sec:stmtsemantics} that the semantics for formulas in statements
are non-blocking.
The reason we opted for this approach, was that we preemptively knew that dealing with quantifiers could be problematic.
We want to make the code extractable to OCaml, and this means we cannot rely on the runtime of why3 and hence
we would need to implement some algorithm to deal with quantified proofs.
This was not an objective of this project, and thus we chose a non-blocking semantics.

Possible solutions for implementing a blocking semantics are to either have two different assertion languages,
one for executable assertions, i.e. user-specified assertions to be evaluated during runtime,
and another language for
assertions used in the verification condition generation to strengthening the precondition.
In the executable assertions we do not allow quantifiers, but all other logical constructors would be included.
In the assertions for VC we would include all the constructs.

Another solution would be to simulate blocking using the predefined constructors.
This means we would use boolean expressions for assertions in statements, as the expressiveness of boolean expressions are equivalent to the assertion language without quantification,
and any assertion which should fail can be simulated using an unbound variable.
For instance, we could formalize the semantic rules for assertions in statements as follows:

\begin{lstlisting}[language=sml]
| ESAssert     : forall f, st : store 'a.
                 eval_bexpr f st (Enormal true) ->
                 eval_stmt (Sassert f) st (Enormal st)
| ESAssert_err1: forall f, st : store 'a.
                 eval_bexpr f st (Enormal false) ->
                 eval_stmt (Sassert f) st (Eabnormal Eassert)
| ESAssert_err2: forall f, st : store 'a.
                 eval_bexpr f st (Eabnormal err) ->
                 eval_stmt (Sassert f) st (Eabnormal err)
\end{lstlisting}
where \texttt{Eassert} is a new error for failing assertions, that we would need to add to
the set of valid errors.

In the evaluator this could be implemented as follows:
\begin{lstlisting}[language=haskell]
| Sassert b -> if b then () else aeval x; ()
\end{lstlisting}
where the variable $x$ is unbound, thus provoking an unbound error, ensuring that the program
cannot be proven correct.


% - We dont have a good exportable way to deal with quantifiers, a possible solution
% could be to have two different assertion languages, one for blocking and one for reasoning about
% verification conditions.

% bexpr kan udtrykke det samme som formulas uden kvantorer og any blocking can be simulated with an
% unbound variable.


\subsection{Evaluator}\label{sec:evaluator}
With our formalization of the semantics, we can define an evaluator and must do so if we want an extractable evaluator for WHILE.
We showed earlier how, the total function lemmas got folded out and expressed
the total function satisfying the semantics.
The evaluator essentially implements this approach, although modelling the store as a mapping is not satisfactory.
Firstly, the semantics for statements results in a store under normal evaluation
and for an evaluator which returns a mapping to be useful we want the be able to
reason about the final store.
This can be done by mappings, but requires any post processing to have knowledge
of the variables assigned in the program either by the user or computationally.
An easier approach would be to have a data-structure which stores key value pairs.
We ended up using a mutable list of key value pairs, which has some promises about its state.



\subsubsection{Modelling a store}\label{sec:model}
To simulate the store, we had to implement a seperate module \texttt{ImpMap},
since we ran into a number of roadblocks using the predefined data-structures in the standard library.
These will be explained further in \ref{sec:results}.
\\~\\
We define the store through a record called ``state''.
This record contains two different fields.
Firstly we have \texttt{lst} which is a linked list of a key value pair,
whilst the second field \texttt{model} is a mapping of the same type as the state used for defining the semantics.
Notice here that the model is marked with the ghost keyword. This means that the model can only be used in a logical context
and therefore cannot manipulate any computations.
We use the model for reasoning about the store through the verification conditions.

The module implements three functionalities, \texttt{empty}, \texttt{add} and \texttt{find}.

Instantiation of a state by \texttt{empty} should ensure that the image of the store is \{\texttt{None}\}.

You can add key value pairs to a state using \texttt{add}, which will add the pair to the existing mappings.

Lastly, \texttt{find} tries to find the key in the list and if it does not exist we throw an error.

This use of the model enables us to easily propagate the error throughout the evaluation.
The full implementation can be seen in Appendix \ref{codemodel}.

% \begin{lstlisting}[caption={Model for a store},label={lst:why3state},language=sml]
% type model_t = M.map int (option int)

% predicate match_model (k: int) (v: int) (m : model_t) =
% match M.get m k with
% | None -> false
% | Some v' -> v = v'
% end

% function helper (m : model_t) (pair: (int,int)) : bool =
% let (k,v) = pair in match_model k v m

% type state = { mutable lst : list (int, int);
%                ghost mutable model : model_t
%              }
%       (* invariant { for_all (helper model) lst } *)
%       (* by { lst = Nil ; model = (fun (_ : int) -> None) } *)

% exception Unbound

% function domain (s : state) : M.map int (option int) = s.model

% let function empty () : state
% ensures {forall k. M.get result.model k = None }
% = { lst = Nil ; model = (fun (_ : int) -> None) }

% let add (k: int) (v: int) (s : state) : ()
% writes { s.lst }
% writes { s.model }
% ensures { s.model = M.((old s.model)[k <- Some v]) }
% ensures { hd s.lst = Some (k,v) /\ match_model k v s.model }
% = s.lst <- (Cons (k,v) s.lst);
%   s.model <- M.(s.model[k <- Some v])

% let rec find (k: int) (s : state) : int
%   variant { s.lst }
%   ensures { match_model k result s.model}
%   raises { Unbound -> M.get s.model k = None}
% = match s.lst with
%   | Cons (k', n) s' -> if andb (k <= k') (k' <= k) then n
%      else find k { lst = s'; model = s.model}
%   | Nil -> raise Unbound
%   end
% \end{lstlisting}


\subsubsection{Evaluation}
For the evaluation we define the imperative store, and four functions, one for each construct,
arithmetic epressions, boolean expressions and statements, and a ``top-level'' function which evaluates
evaluates a statement and extracts the result of evaluation, and also clearing the store.

In the contract of each function, we describe both what happens under normal execution and what happens under abnormal execution, using the \texttt{ensures} and \texttt{raises}, for instance we have:

\begin{lstlisting}[caption={Evaluation of arithmetic expression},label={lst:why3aeval},language=sml]
let rec aeval (a: aexpr id) : int
  variant { a }
  ensures { eval_aexpr a (IM.domain sigma) (Enormal result) }
  raises { IM.Unbound -> eval_aexpr a (IM.domain sigma) (Eabnormal Eunbound) }
= match a with
  | Acst i -> i
  | Avar v ->  IM.find v sigma
  | ABin a1 op a2 -> (eval_op op) (aeval a1) (aeval a2)
end
\end{lstlisting}

\autoref{lst:why3aeval} present the evaluation of arithmetic expressions. Like for the lemma about the semantics
being a total function, we want the evaluator to adhere to the semantics from the inductive predicate.
In the case of an unbound variable the \texttt{Imperative Map find} will raise an exception, in this case
we will ensure that the behaviour of the evaluation is Eabnormal Eunbound.
It should be clear, that this can easily be extended to other errors, by doing a conjunction of
all possible (exception $\rightarrow$ semantical behaviour) pairs.
We follow the same structure for evaluation of boolean expressions and statements.
The vc is easily proved for arithmetic and boolean expressions.
Again we cannot show it for statements, neither for total nor partial correctness.

\subsection{Verification Condition Generation}\label{sec:vcg}
The other major part of the projet is the formalisation of an extractable verification condition generator.
in \autoref{sec:impwlp} we will introduce the formalisation of the weakest liberal precondition,
but before we can do so, we need to address the matter of variables in formulas.


\subsubsection{Variable substitution}
From \autoref{TODO} it should be clear that there are multiple situations where we want to substitute
all free occurences of a variable by an new variable.
For instance when $s$ is $x := a$ in $wlp(s, Q)$ where $Q$ is a formula, and $s$ is the assignment of $a$ to $x$,
we want to update each occurence of $x$ in $Q$, but do not substitute bound instances of $x$.
The rules are as follows, we only consider cases where variables might occur. $f_{i}$ is formulas, $x,y,z$ are variables and $b_{i}, a_{i}$ describes boolean expression and arithmetic expressions respectively. Conjuction for booleans and formulas are equivalent.

\begin{align*}
\subst{x}{x}{y} &= y \\
\subst{z}{x}{y} &= x \\
&\\
\subst{z}{x}{y} &= x \\
\subst{(a_{1} \leq a_{2})}{x}{y} &= \subst{a_{1}}{x}{y} \le \subst{a_{2}}{x}{y} \\
\subst{(b_{1} \wedge b_{2})}{x}{y} &= \subst{b_{1}}{x}{y} \wedge \subst{b_{2}}{x}{y} \\
\subst{(\neg b)}{x}{y} &= \neg\subst{b}{x}{y} \\
&\\
\subst{(f_{1} \Rightarrow f_{2})}{x}{y} &= \subst{f_{1}}{x}{y} \Rightarrow \subst{f_{2}}{x}{y} \\
\subst{(\forall x. f)}{x}{y} &=  \forall x. f\\
\subst{(\forall z. f)}{x}{y} &=  \forall z. \subst{f}{x}{y}\\
\end{align*}
The $subst$ must also adhere to the rule that $y$ must be free for $x$ in whatever the expression/formula,
meaning a free occurence of $x$ must not be bound when substituted by $y$.

From the formulation of the substitution function it should be easy to see, that we recurse down the syntax tree.
The most interesting case is when we meet a quantifier. If the variable $x$ we want to substitute is bound then we stop recursion, as all occurences of $x$ will then be bound.

It should be noted that we do not ensure that $x$ must be free for $y$,
thus the function is rather unsafe.
We only ever use it in a context where this cannot happen, since we instantiate a new variable that does not occur in the formula, bound or free. The variable is so-called fresh in the formula.

The way we generate a fresh variable, is by traversing the syntax tree of the formula and recursively taking the maximal variable value and adding 1.
This works since we use integers as identifiers.
A similar result can be obtained by all infinite enumerable sets. For instance if identifiers were strings, one could add a new character to the longest variable name.

This seems rather trivial, however we formulated predicates that state whether a variable is fresh in an expression or formula. We then afterward tried making a lemmas which stated:

\begin{lstlisting}
lemma fresh_var_is_fresh : forall f v.
    v = fresh_from f -> fresh_in_formula v f
\end{lstlisting}

The lemma simply asserts that generating a fresh variable $v$ from $f$ implies that $v$ is fresh in $f$.
We got the inspiration for this approach from \cite{TODO: WP revisited}.
In their work they formulate the fresh_from function using axioms.
That is they provide a function declaration and then the function ``computes'' the value based on a set of axioms.
For us this is not a viable approach, as such functions cannot be extracted. Therefore we defined the function ourself and stated axiom as a lemma to ensure that \texttt{fresh\_from} always ensures a fresh variable.
Although this seems like a trivial result, we were not able to prove this using the SMT solvers.
We tried to define this system in isabelle to see how difficult it was to prove using a proof assistant, where one has more control. We succeeded in proving this.
But since, this does not entail that there might be a bug in the why3 code.
The isabelle proof might not be the cleverest way to prove it, but we proved it the following way.
let $V$ be the set of all vars in an arithmetic expression $a$.
Then assuming $v$ is the maximum identifier in $a$, we show that $\forall x \in V. v \ge x$. We do so by induction.
By this we can then show $v + 1 \notin V$.
We then made a lemma stating that given $v \notin V$ then $v$ is fresh in $a$.
From this we can show that $v + 1$ is fresh in $V$.
We then do the same for boolean expression and formulas, and the proof is done.
The full proof can be seen in \autoref{TODO appendix}.

\subsubsection{Weakest Liberal Precondtion}\label{sec:impwlp}
\paragraph{Implementation of Weakest Liberal Precondition calculus.}
We implemented the rules for weakest liberal precondition by a recursive function that directly
follows the inference rules.
We focus on two of the rules, namely assignments and while statements, which both does substitution in the formula.
For assignments we define the rule as follows:

\begin{lstlisting}
| Sass x e -> let y = fresh_from q in
              Fall y (Fimp (Fand (Fterm (Bleq (Avar y) e))
                                 (Fterm (Bleq e (Avar y)))) (subst_fmla q x y))
\end{lstlisting}

We first generate a fresh variable, which is then substituted into the postcondition $Q$.
We further follow the semantics, however since we do not have equality we use $y \leq e \wedge e \leq y$.

The other interesting rule, for while, is implemented as follows:
\begin{lstlisting}
| Swhile cond inv body ->
  Fand inv
       (abstract_effects body
       (Fand (Fimp (Fand (Fterm cond) inv) (wp body inv))
             (Fimp (Fand (Fnot (Fterm cond)) inv) q))
       )
\end{lstlisting}

Here the interesting thing is the function \texttt{abstract\_effects}. This function quantifies over all assigned variables in the body of the loop and substitute the quantifiers with the free variables in the formula
$((cond \wedge inv) \Rightarrow wp(body,inv)) \wedge ((\neg cond \wedge inv) \Rightarrow Q)$. like \texttt{fresh\_from} this function is inspired by \cite{TODO}. The story is the same. They used axioms to define the function, whereas we have to implement it.
\texttt{abstract\_effects} takes a statement $s$ and formula $f$.
If the statement is an assignment
a fresh variable is made and substituted into $f$, and we then quantify the freshly made variable.
This is not the most efficient solution, as multiple assignment to the same variable will create unused quantifiers.
All the cases will traverse the abstract syntaxt tree or is a leaf and does nothing.

\begin{lstlisting}
let rec function abstract_effects (s : stmt int) (f : formula int) : formula int
  variant { s }
= match s with
  | Sskip | Sassert _ -> f
  | Sseq s1 s2 | Sif _ s1 s2 -> abstract_effects s2 (abstract_effects s1 f)
  | Sass x _ -> let v = fresh_from f in
                let f' = subst_fmla f x v in
                Fall v f'
  | Swhile _ _ s -> abstract_effects s f
  end
\end{lstlisting}

Whilst we implemented this function directly, we still want the properties for the function to hold.
\parapgraph{specialization.} Firstly, we have

\begin{lstlisting}
  lemma abstract_effects_specialize : forall st : store int, s f.
    eval_formula (abstract_effects s f) st -> eval_formula f st
\end{lstlisting}

which states that evaluation of applying abstract effects off s on f in state st implies that f evaluates to true in st. Essentially it states that if we quantify the variables in f and the formula is true under the quantification, then $f$ also hold if the variables are not quantified.
THIS DOES NOT MAKE ANY SENSE??? PLEASE LOOK AT IT TOMORROW.

\paragraph{Quatification over conjunction.}
Secondly, we have
\begin{lstlisting}
lemma abstract_effects_distrib_conj : forall s p q st.
   eval_formula (abstract_effects s p) st /\ eval_formula (abstract_effects s q) st ->
   eval_formula (abstract_effects s (Fand p q)) st
 \end{lstlisting}

Which states that if we apply abstract effects of s on two formulas p and q and they both evaluates to true,
then evaluating the result of applying the abstract effects of s over the conjunction of p and q, must also be true.

\paragraph{Monotonicity.}
Thirdly, we have the property of monotonicity.

\begin{lstlisting}
lemma abstract_effects_monotonic : forall s p q.
   valid_formula (Fimp p q) ->
   forall st. eval_formula (abstract_effects s p) st ->
   eval_formula (abstract_effects s q) st
 \end{lstlisting}

Essentially what monotonicity states, is that applying additional assumptions to a formula will not change the meaning. It should be noted that we want to quantify all states in the entailment, as the property does not hold for a fixed state.

\paragraph{Invariance.}
Lastly, we consider the notion of invariance.
TODO: I DONT THINK THIS IS NECESSARY.

In \cite{TODO} the axiomatized version of the following lemma, is used to define which variables should be quantified.
We explicitly state this in the body of \texttt{abstract\_effects},
but for good measure we include it as a lemma. The property states that if a the formula $q$ abstracted by $s$ is true in some state, then the weakest precondition on the same abstraction on $q$ by $s$ is true should also be true. When \cite{} uses this as an axiom, it ensures, that \texttt{abstract\_effects}

\paragraph{Proofs of properties for abstract\_effects}
We have not been able to automatically prove these lemmas in why3.
At the moment we have a partial proof for the function in Isabelle.
???? WRITE SOME MORE ????

\paragraph{Properties and Soundness of WLP}
One of the main goals in making a formally verified Verification condition generator is to ensure the correctness of the implementation.
We consider the correctness through its soundness.
To do so we must consider two of the same properties we just stated for a\texttt{abstract\_effects}, namely monotonicity and conjunction distribution.

Monotonicity of WLP is that if for two formulas $p$ and $q$ the formula $\vDash p \Rightarrow q$ then $\vDash wp(s, p) \Rightarrow wp(s,q)$. Notice again that this must hold for all statements and states.
Distribution of weakest precondition over conjuction, is similar to the lemma of abstract effects only the transformation on $p$ and $q$ are now considered for $wp$.

\begin{lstlistings}
lemma monotonicity: forall s p q.
      valid_formula (Fimp p q) -> valid_formula (Fimp (wp s p) (wp s q))

lemma distrib_conj: forall s sigma p q.
      eval_formula (wp s p) sigma /\ eval_formula (wp s q) sigma ->
      eval_formula (wp s (Fand p q)) sigma
\end{lstlistings}

Once again we, we not able to directly show this. We tried to unfold the recursion on $s$, in a similar manner to how we proved determinism.
We did not achieve anything by this.
We have a formalized proof of both properties in Isabelle.
The proof for monotonicity can be seen in \cite{lst:isamono}.
As mentioned we prove the lemma by induction on $s$.
And mark $p$ and $q$ as arbitrary, since the lemma should hold for any non-fixed $p$ and $q$.
The proof only shows the cases for sequences, assignments and while.
The case for sequences are actually directly proved from the assumptions.
We only distinguish this case because we need to simplify the other trivial cases.
For both assignment and while we used the sledgehammer to find the proofs by metis,
which is a complete automatic theorem prover for first order logic with equality\cite{TODO sledgehammer paper}.

\begin{lstlisting}[caption={Proof of monotonicity in Isabelle},label={lst:isamono},language=sml]
lemma monotonicity : "valid_formula (FImp p q) \<Longrightarrow>
  valid_formula (FImp (wp s p) (wp s q))"
proof(induction s arbitrary: p q)
  case (SSeq s1 s2)
  then show ?case by auto
next
  case (Sassign x1 x2)
  then show ?case
    by (metis abstract_effect_writes abstract_effects.simps(2)
         eval_formula.simps(2) valid_formula_def wp.simps(3))
next
  case (Swhile x1 x2 s)
  then show ?case
    by (metis abstract_effect_writes abstract_effects.simps(2)
        eval_formula.simps(2) valid_formula_def wp.simps(3))
qed (simp_all add: valid_formula_def)
\end{lstlisting}

For distribution over a conjuction, the proof is straight forward.
We do induction with the same setup.
Again the sequences, assignments and while cannot trivially be proved by simplification,
but the same metis proof used for monotonicity can be used for all three cases.

With these two properties, we should be able to show soundness for the wp function.
We decided to not prove the completeness of the function because this says something about the expressiveness of the function, whereas it is more important to ensure that
the function is correct.
Using soundness to show correctness is two-fold.
On the one hand showing the soundness of $wp$ ensures that $\hoare{wp(s,Q)}{s}{Q}$ is valid for partial correctness for all $s$ and $q$, ensures that we cannot generate invalid verification conditions.
On the other hand, we already know weakest precondition to be sound and proving it for the function $wp$ ensures that our implementation adheres to the semantics or atleast an equivalent semantic.

We have not gotten to show this.
TODO why have we not done this?????

\section{Extraction of code}\label{sec:extract}
Why3 makes it possible to extract whyML code to either Ocaml or C code.
One of the main goals of this project was to make the verified code extractable.
To achieve this we had to adhere to some limitations.
Mostly these are related the expressiveness and hierachy of whyML.

\paragraph{Extraction is correct by construction.}
The extraction of whyML programs is correct by construction.
This means that each syntactical object is directly translated into an equivalent object
in the target language. For instance a program function written in whyML denoted by either \texttt{let}, \texttt{let function}, \texttt{let rec function} or \texttt{let rec} directly translate into its equivalent Ocaml function. This ensures that extracted code does the same when exported.
On the downside this means that we can only extract our code to Ocaml, but not to C, since there is no
language defined notion of Abstract Data Types.
Thus our choice to represent our object language as ADT's have limited our ability to extract code.
However structuring the code in C friendly manner would likely become quite tedious.

\paragraph{Program functions and Logical functions.}
Just as the object level limits what we can do in terms of extracted code, so is the logical level.
There is a clear distinction between logical functions and program functions in why3.
All program functions is specified with a \texttt{let} and can be extracted,
while the logical functions can be used to reason about the program functions.
Actually extracting the code has been a bit of a challenge.
Firstly, as mentioned earlier we compromised in regards to assertions in the operational semantics of the program. For the reason that it is hard to argue about logical quantifiers on a program level, while it is much easier to do on a logical level. Hence why evaluation of Formulas is done with a predicate and not a program function.
Furthermore, making the actual evaluator was quite troublesome.
For defining the actual semantics we used inductive predicates, which is a logical construct, and this is extremely useful because we can argue the correctness of evaluation of a statement on a program level.
In this reasoning we used maps, which is simply functions, with some syntactical sugar for updates and computations.
The problem arises when we need to define the environment for the actual interpreter.

We were not able to find any module which was extractable and would adhere to the same logical meaning as that of maps.
Hence we had to implement the module explained in \ref{sec:env}.
And while there might be better ways to handle the store, Our current approach seems to suffice.

In regards to the verification condition generation, we did not run into too much trouble.
As mentioned previously our approach to the weakest precondition generation follows the same structure as WP revisited in why3\cite{},
but deviates in how functions are defined.
Axiomatized functions can clearly not be exported,
since they dont have a function body but simply must adhere to a set of axioms.
And therfore we need to prove explicitly that our function is correct,
whereas axiomatization will be correct by definition.
But actually extracting the $wp$ function is rather simple, as it essentially just transforms an ADT.
And clearly can be defined in on the program level.


\paragraph{Using the extracted code.}
The code is essentially split into two different functionalities.
The evaluator does imperative evaluation on some statement in the mutable environment.
We can extract the code for evaluation by the following command:

\begin{lstlisting}
why3 extract --recursive -D ocaml64 -L . evaluator.mlw -o eval.ml
\end{lstlisting}

This will tell to recursively add dependencies into the module defined in the evaluator.mlw file, using the ocaml64 driver and create a module in the file eval.ml.
We can then use the module in an ocaml project (where we might define a parser etc. for the language).
\autoref{lst:ocamlexample} shows an example of a very simple program.
The program, defines two WHILE programs, which are very simple.
\texttt{prog} is a simple assignment. One thing to note is that because we used the int type in why3, we have to use the \(Z.of\_int\) because int's in why3 are unbounded integers from the zarith library.
\texttt{prog'} defines a slightly more complicated program, which does 2 assignments, and the second assignment tries to assign an expression with an unbound variable to a variable.

\begin{lstlisting}[caption={ocaml program using the evaluator},label={lst:ocamlexample},language=sml]
open Eval
let prog =
             Sass  (Z.of_int 1, ABin (
                        Acst (Z.of_int 5)
                        , Mul
                        , Acst (Z.of_int 10)))
let prog' = Sseq (
             Sass  (Z.of_int 1, ABin (
                        Acst (Z.of_int 5)
                        , Mul
                        , Acst (Z.of_int 10)))
             , Sass (Z.of_int 2 , ABin (
                         Acst (Z.of_int 5)
                       , Add
                       , Avar (Z.of_int 3))))
let res = eval_prog prog
let () = List.iter (fun (k,v) -> Format.printf "%d, %d\n" (Z.to_int k) (Z.to_int v)) res
let () = List.iter (fun (k,v) -> Format.printf "%d, %d\n" (Z.to_int k) (Z.to_int v)) sigma.lst
let res' = eval_prog prog'
\end{lstlisting}

Compiling the program with

\begin{lstlisting}
\textdollar ocamlbuild -pkg zarith main.native
\end{lstlisting}

and running it by:

\begin{lstlisting}
\textdollar ./main.native
\end{lstlisting}

The following output is produced:

\begin{lstlisting}
1, 50
Fatal error: exception Eval.Unbound
\end{lstlisting}

The evaluation of the first program is that variable 1 is 50. The second List.iter prints nothing,
this shows that we correctly resets the environment. Lastly \texttt{prog'} correctly produces an Unbound exception.
\\~\\
The procedure for the verification condition generator is the same.
It is worth noting that we have not implemented any way to use the generated formula for anything.
And thus it may render useless for now to extract it.
In case one want to actually verify the correctness of a WHILE program, we can use the povers of Why3 and use the meta level of logic using a goal.
Take the following program

\begin{lstlisting}[caption={WHILE program which multiples q and r by repeated addition},label={lst:whileexample},language=sml]
q = 10;
r = 5;
res = 0;
ghost = q;
while (q > 0)
invariant {(res $\leq$ (ghost - q) * r && (ghost - q) * r $\leq$ res) /\ 0 \leq q} {
      res = res + r;
      q = q - 1;
};
assert {res $\leq$ ghost * r /\ ghost * r $\leq$ res};
\end{lstlisting}

The syntax should be straight forward. The program will define 4 variables, $q,r,res$ and $ghost$.
The program will execute the while loop $q$ times. In the body of the loop we add $r$ to $res$ and decrement $q$ by 1. We keep as invariant that $res = (ghost - q) * r$. Lastly we assert that $res$ is the orignal value of $q$, described by $ghost$, multiplied with $r$.
Writing this as an AST, we have:

\begin{lstlisting}[caption={WHILE AST in why3},label={lst:whileast},language=sml]
let function prog () =
  (* var 2 is q *)
  (Sseq (Sass 2 (Acst 10))
  (* var 1 is res *)
  (Sseq (Sass 1 (Acst 0))
  (* var 0 is r *)
  (Sseq (Sass 0 (Acst 5))
  (* Ghost var = q *)
  (Sseq (Sass (-1) (Avar 2))

  (Sseq (Swhile (Bleq (Acst 0) (ABin (Avar 2) Add (Acst 1)))
        (* res <= (ghost - q) * r /\ res >= (ghost - q) * r *)
        (Fand (Fterm (Band (Bleq (Avar 1)  (ABin (ABin (Avar (-1)) Sub (Avar 2)) Mul (Avar 0)))
             (Bleq (ABin (ABin (Avar (-1)) Sub (Avar 2)) Mul (Avar 0)) (Avar 1))))
        (* 0 <= q *)
             (Fterm (Bleq (Acst 0) (Avar 2))))
        (* NOW BODY OF THE LOOP *)
        (Sseq (Sass 1 (ABin (Avar 1) Add (Avar 0)))
              (Sass 2 (ABin (Avar 2) Sub (Acst 1)))))

        (* Assert {res = ghost * r} *)
        (Sassert (Fterm (Band (Bleq (Avar 1) (ABin (Avar (-1)) Mul (Avar 0)))
        (Bleq (ABin (Avar (-1)) Mul (Avar 0)) (Avar 1))))))))))
      \end{lstlisting}

We can then construct a goal checking if the weakest precondition of this program is valid.

\begin{lstlisting}
goal mult_is_correct :
   valid_formula (wp (prog ()) (Fterm (BCst true)))
\end{lstlisting}

Feeding this to an SMT solver such as Eprover, we can show the program to be correct, in around half a second.

