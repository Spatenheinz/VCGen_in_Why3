\section{Implementation}\label{sec:implementation}
In this section we describe how we implemented the presented syntactical and semantical system
in Why3. This is presented for each syntactical component: arithmetic expressions, boolean expression, statements, and formulas. In \ref{sec:evaluator} we describe how we implemented
an evaluator for the language, and in \ref{sec:vcg} the implementation of verification
condition generation is presented.

For each component we define a set of properties which must hold for a correct implementation.
We sometimes accompany the Why3 implementations with proofs conducted in Isabelle, to reason
about parts that were tricky in Why3.

The entire implemetation can be seen in \cref{code}.

\subsection{Syntax}
\section{Code}\label{code}
\subsection{Syntax}
\lstinputlisting{../code/syntax.mlw}

\subsection{Semantics}
\lstinputlisting{../code/semantics.mlw}

\subsection{Formula}\label{codeformulas}
\lstinputlisting{../code/formula.mlw}

\subsection{Fresh}
\lstinputlisting{../code/fresh.mlw}

\subsection{Subst}
\lstinputlisting{../code/subst.mlw}

\subsection{Weakest Precondition}
\lstinputlisting{../code/wp.mlw}

\subsection{Imperative Environment}\label{codemodel}
\lstinputlisting{../code/dict.mlw}

\subsection{Evaluator}
\lstinputlisting{../code/evaluator.mlw}

\subsection{Isabelle code}
\lstinputlisting{../code/Isabelle/aexpr.thy}
\lstinputlisting{../code/Isabelle/bexpr.thy}
\lstinputlisting{../code/Isabelle/formula.thy}
\lstinputlisting{../code/Isabelle/stmt.thy}


\subsection{Arithmetic Expressions}
In the following we will describe how the semantics for arithmetic expressions are implemented,
and which properties we want to hold to ensure correctness of the implementation.

\subsubsection{Semantics}
In \autoref we defined the big-step semantics of WHILE.
In both why3 and Isabelle it is possible to create inductive predicates.
An inductive predicate is simply a predicate with a set of proof constructors defined inductively.
This is a great tool to make a close to one to one mapping from the pen and paper version and a formalized
version.
We define the semantics of arithmetic expressions as follows:
\begin{lstlisting}[caption={semantics of arithmetic expressions},label={lst:why3aexpr},language=sml]
type e_behaviour 'a = Enormal 'a | Eabnormal

inductive eval_aexpr (aexpr 'a) (store 'a) (e_behaviour int) =
  | EACst : forall n, s : store 'a.
             eval_aexpr (Acst n) s (Enormal n)
  | EAVar : forall x n, s : store 'a.
             s[x] = Some n ->
             eval_aexpr (Avar x) s (Enormal n)
  | EAVar_err : forall x, s : store 'a.
             s[x] = None ->
             eval_aexpr (Avar x) s Eabnormal
  | EASub : forall a1 a2 n1 n2, s : store 'a, op.
             eval_aexpr a1 s (Enormal n1) ->
             eval_aexpr a2 s (Enormal n2) ->
             op = Sub ->
             eval_aexpr (ABin a1 op a2) s (Enormal (n1 - n2))
  | EAMul : forall a1 a2 n1 n2, s : store 'a, op.
             eval_aexpr a1 s (Enormal n1) ->
             eval_aexpr a2 s (Enormal n2) ->
             op = Mul ->
             eval_aexpr (ABin a1 op a2) s (Enormal (n1 * n2))
  | EAOp_err1 : forall a1 a2, s : store 'a, op.
             eval_aexpr a1 s Eabnormal ->
             eval_aexpr (ABin a1 op a2) s Eabnormal
  | EAOp_err2 : forall a1 a2, s : store 'a, op.
             eval_aexpr a2 s Eabnormal ->
             eval_aexpr (ABin a1 op a2) s Eabnormal
\end{lstlisting}

From \autoref{lst:why3aexpr} it should be straight forward to see that each of the predicates, corresponds to
each of the inference rules in \autoref{}.
Each subterm before \texttt{->} is the premises of the specific inference rule and the last term is the conclusion.
There is however a slight difference.
Initially we split the EAOp rule into an EAMul and EASub, since we though this would be a necessity,
and more closely resembles how semantics usually are represented in text books and as such should be considered independently.
However we found that this indeed is not necessary because of the rather trivial and mechanized reasoning about such binary operators and in fact reduces complexity of the automated proofs as is apparent in \autoref{fig:aexpr_props}.
The figure shows the 3 goals for arithmetic expressions. For a description of each property see \autoref{TODO}.
In both versions of the semantics the goals are Valid, however we see a big difference in the steps it takes to validate it.
For the initial version the steps for Goal 1 and Goal 3 takes roughly 2.5 as many steps as the current version. For Goal 2 the initial version requires 4 times as many steps as the current.

\begin{figure}
\begin{lstlisting}
INITIAL VERSION (version where we specifically distingush the two cases)
File semantics.mlw:
Goal eval_aexpr_total_fun'vc.
Prover result is: Valid (0.19s, 4170 steps).

File semantics.mlw:
Goal eval_aexpr_fun'vc.
Prover result is: Valid (0.58s, 14543 steps).

File semantics.mlw:
Goal eval_aexpr_total'vc.
Prover result is: Valid (0.03s, 428 steps).

NEW VERSION (Where we convert aop to a why3 binary operator).
File semantics.mlw:
Goal eval_aexpr_total_fun'vc.
Prover result is: Valid (0.09s, 1577 steps).

File semantics.mlw:
Goal eval_aexpr_fun'vc.
Prover result is: Valid (0.20s, 3449 steps).

File semantics.mlw:
Goal eval_aexpr_total'vc.
Prover result is: Valid (0.01s, 105 steps).

\end{lstlisting}
\caption{Steps to prove properties for arithmetic expressions}
\label{fig:aexpr_props}
\end{figure}

The new and improved semantic representation makes uses of a function \texttt{eval\_op :: axpr -> (int -> int -> int)} and used it in the conclusion as \texttt{eval\_aexpr (ABin a1 op a2) s (Enormal ((eval\_op op) a1 a2)}.
Hence we get a direct correspondence to the semantics presented in \autoref{fig:aexprsemantics}, by removing EAMul and EASub and replacing it with

\begin{lstlisting}
let function eval_op (op : aop) : (int -> int -> int) =
    match op with
    | Mul -> (*) | Sub -> (-)
    end
...

  | EABin : forall a1 a2 n1 n2, s : store 'a, op.
             eval_aexpr a1 s (Enormal n1) ->
             eval_aexpr a2 s (Enormal n2) ->
             eval_aexpr (ABin a1 op a2) s (Enormal ((eval_op op) n1 n2))
\end{lstlisting}


Notice furthermore we can define the behaviour of an operation as a Discriminated union of either a polymorphic \texttt{Enormal} behaviour or an \texttt{Eabnormal error},
we do so to reuse the same types for arithmetic and boolean expressions along with statements.
In the current implementation the only way to not result in normal behaviour is when we have an unbound variable.

If we in the future were to add additional errors we let Eabnormal hold an ADT containing different errors.
We tried looking into this matter, by adding division as an operator and adding additional semantics,
with possibility of resulting in an \texttt{Ebnormal Ediv0} error. We however did not find a way to prove this,
as the SMT solvers would time out on the properties.
Again this is probably from the exponential growth of the size of the formula.

% The equivalent predicate in Isabelle is nearly identical except for a few syntactical differences, for
% instance we dont have to explicitly instantiate each variable as they implicitly instantiated to be fixed but arbitrary.

% \begin{minted}
% datatype 'ty behaviour = Enormal 'ty | Eunbound

% fun eval_binop :: "aop ⇒ (int ⇒ int ⇒ int)" where
%   "eval_binop Mul = (*)"
% | "eval_binop Sub = (-)"

% inductive eval_aexpr :: "int aexpr ⇒ int state ⇒ int behaviour ⇒ bool" where
%     EACst     : "eval_aexpr (Cst n) s (Enormal n)"
%   | EAVar     : "s x = Some n ⟹
%                  eval_aexpr (Var x) s (Enormal n)"
%   | EAVar_err : "s x = None ⟹
%                  eval_aexpr (Var x) s Eunbound"
%   | EABin     : "eval_aexpr a1 s (Enormal n1) ⟹
%                  eval_aexpr a2 s (Enormal n2) ⟹
%                  eval_aexpr (BinOp a1 op a2) s (Enormal ((eval_binop op) n1 n2))"
%   | EABin_err1 : "eval_aexpr a1 s Eunbound ⟹
%                   eval_aexpr (BinOp a1 op a2) s Eunbound"
%   | EABin_err2 : "eval_aexpr a1 s (Enormal n) ⟹
%                   eval_aexpr a2 s Eunbound ⟹
%                   eval_aexpr (BinOp a1 op a2) s Eunbound"
% \end{minted}

% The interesting thing here, is that we use the other proposed semantics, the one which follows the rules directly.
% The reason for this distiction is because while we can prove certain properties in why3 with one semantic
% and certain properties in Isabelle. For instance, if we take a look at determinism for arithmetics expressions,
% which we introduced in \autoref{}


\subsubsection{Properties}\label{sec:aprops}
The following paragraphs describes the properties we want to hold for arithmetic expressions. They are determinism and totality.

\paragraph{Determinism.}
We want our program to be deterministic which means each part of the semantics must be deterministic, i.e. the semantic relation must be functional.

To show that a judgement or inductive relation of an arithmetic expression is functional, we have

\[
  \judge{a,\sigma}{n} \; \text{and also} \; \judge{a,\sigma}{n'} \; \text{then} \; n = n'
\]

We can formalize this very easily in Why3 using a lemma, as shown in \autoref{lst:why3funnorec}.

\begin{lstlisting}[caption={lemma for functional arithmetic expressions},label={lst:why3funnorec},language=sml]
lemma eval_aexpr_fun_cannot_show : forall a, s : store 'a, b1 b2.
   eval_aexpr a s b1 -> eval_aexpr a s b2 -> b1 = b2
\end{lstlisting}

The problem arises when we try to discharge this proof obligation to an SMT solver.
It seems like this cannot directly be proven, and the reason can be found in the usual way such formalism is proved for ASTs.
More specifically, we do so by induction, and lemmas in why3 has no notion of induction.
It is possible to dotransformations on a lemma, and a strategy such as \texttt{induction_pr} would correspond to this induction.
Doing such a transformation we split the lemma into 6 subgoals, one for each inductive predicate rule, which can then be discharged.
The first three goals that corresponds to \texttt{EACst}, \texttt{EAVar} and \texttt{EAVar\_err} can be proven by Alt-Ergo, the remaining cases are for binary operators and these cannot be shown by an SMT solver.
In this case we would want to use either Coq or Isabelle. We tried opening the proofs in both Isabelle and Coq.
The Isabelle driver simply does not work, giving the following error:
\\~\\
\texttt{There is no verification condition "eval\_aexpr\_fun\_cannot\_show".}
\\~\\
This suggest that we cannot reason about lemmas using Isabelle.
Opening the proof in Coq allows us to perform usual reasoning in Coq, but with our limited knowledge of Coq, we quickly had to abandon this approach.
\\~\\
Instead we tried a different approach. Although why3 does not allow for induction, it is possible to define
recursive lemmas. A recursive lemma differs from a regular lemma, in that it looks more like a ``function''.
We can define \texttt{eval_aexpr_fun_cannot_show} as a recursive lemma as seen in \autoref{lst:why3funrec}.
The lemma is defined by two parameters: 1) the arithmetic expression and 2) a state.
Recursive lemmas are proved through a Verification condition,
hence we need to define a \texttt{variant} to ensure total correctness,
and then we specify the post-condition in the \texttt{ensures}.
The post-condition will serve as the conclusion of the lemma.
This header now corresponds to the lemma in \autoref{lst:why3funnorec}, and the body of the function will then dictate the unfolding of the recursive structure.
Notice here how the result of the body is $()$, as we do not need a result because the logic
of the lemma is defined in the header.

\begin{lstlisting}[caption={Recursive lemma for functional arithmetic expressions},label={lst:why3funrec},language=sml]
let rec lemma eval_aexpr_fun (a: aexpr 'a) (s: store 'a)
    variant { a }
    ensures { forall b1 b2. eval_aexpr a s b1 ->
                            eval_aexpr a s b2 ->
                            b1 = b2
    }
  = match a with
    | Acst _ | Avar _ -> ()
    | ABin a1 _ a2 -> eval_aexpr_fun a1 s; eval_aexpr_fun a2 s
      end
\end{lstlisting}

When discharged, a goal is defined through the verification condition of this lemma, in a similar manner to functions.
After the goal is found (proven or not) the lemma is axiomatised. The specific axiom for \texttt{eval_aexpr_fun} can be seen in \autoref{lst:why3funaxiom}.

\begin{lstlisting}[caption={Axiom of functional lemma},label={lst:why3funaxiom},language=sml]
axiom eval_aexpr_fun :
  forall a:aexpr 'a, s:'a -> option int.
   forall b1:e_behaviour int, b2:e_behaviour int.
    eval_aexpr a s b1 -> eval_aexpr a s b2 -> b1 = b2
\end{lstlisting}

The axiom clearly is equivalent to the lemma. Notice here, that the arguments for the lemma gets universally quantified.

\paragraph{Totality.}
Furthermore we want the semantics to be total, meaning all input have an output, such that we never encounter undefined behaviour.
Formally we have
\[
\forall a \in A, \sigma \in \Sigma. (\exists n. \judge{a, \sigma}{n}) \vee \judge{a, \sigma}{\bot}
\]
which states that for all $a$ and $\sigma$ there either exists an $n$ where $a$ evaluates to $n$ in $\sigma$, or the evaluation of $s$ results in abnormal behaviour. Here $A$ is the set of all arithmetic expressions.

Again we first formalised this as a plain lemma, but as the proof goes by induction on the structure of a,
we had to rewrite this as a recursive lemma. The resulting lemma is shown in \autoref{lst:why3tot}.

\begin{lstlisting}[caption={Axiom of functional lemma},label={lst:why3tot},language=sml]
  let rec lemma eval_aexpr_total (a: aexpr 'a) (s : store 'a)
      ensures {
        eval_aexpr a s (Eabnormal Eunbound) \/
        exists n. eval_aexpr a s (Enormal n)
      }
    = match a with
      | Acst _ | Avar _ -> ()
      | Abin a1 _ a2 -> eval_aexpr_total a1 s; eval_aexpr_total a2 s
      end
\end{lstlisting}

\paragraph{Combination of determinism and totality.}
Another way to state that arithmetic expressions are both total and deterministic is to define a lemma stating that it is an actual total function.
We can do so, again by a recursive lemma, as presented in \autoref{lst:why3totfun}. Notice here that the lemma in this case will simulate the actual
computation of the total function, thus we can use the \texttt{result} keyword in the post-condition.
The post condition will ensure that the total function adheres to the semantics, since the inductive predicate
\texttt{eval_aexpr} should hold for all \texttt{a} and \texttt{s}.

\begin{lstlisting}[caption={Lemma combining totality and determinism for arithmetic expressions},label={lst:why3totfun},language=sml]
let rec lemma eval_aexpr_total_fun (a: aexpr 'a) (s: store 'a)
     variant { a }
     ensures { eval_aexpr a s result }
   = match a with
     | Acst n -> Enormal n
     | Avar v -> match s[v] with
                   | Some n -> Enormal n
                   | None -> Eabnormal Eunbound
                 end
     | ABin a1 op a2 ->
       match eval_aexpr_total_fun a1 s, eval_aexpr_total_fun a2 s with
         | Enormal n1, Enormal n2 -> Enormal ((eval_op op) n1 n2)
         | Eabnormal e, _   -> Eabnormal e
         | _ , Eabnormal e  -> Eabnormal e
       end
     end
\end{lstlisting}


\subsection{Boolean Expressions}
Now we will describe the implementation of semantics for boolean expressions, together with
the properties we want to ensure holds.

The formulation of boolean expressions follows the same pattern as arithmetic expressions.
We define an inductive predicate stating the inference rules in \autoref{TODO}.
For reference see \autoref{TODO}.

Likewise the properties we use for reasoning about correctness of our boolean semantics resemble the properties for arithmetic expressions. Thus we have the following lemmas:

\begin{enumerate}
  \item \texttt{eval\_bexpr\_fun}
  \item \texttt{eval\_bexpr\_total}
  \item \texttt{eval\_bexpr\_total\_fun}
\end{enumerate}

In the formulation of the boolean semantics, we initially encounted some problems in proving our properties.
In the first iteration of the inductive predicate we used the built0in operator \&\&, which works for boolean values.
We then became aware of the function \texttt{andb}, which is defined in the standard library in module bool.Bool.
\texttt{andb} is short-circuitting and presumably the \&\& operator is as well.
From the properties these also
seem the two functions have equivalent semantics.
Interestingly they do not require the same amount of steps in
the proofs. \autoref{tab:stepsbexpr} shows the result of proving the properties with different combinations of
use of the operators.
The columns defines the operator used in the lemma \texttt{eval\_bexpr\_total\_fun},
and the rows defines which operator is used in the inductive predicate.
Each line in a cell corresponds to the number of steps require for Alt-Ergo to prove the lemma.
The lemmas are stated in the order from above.

From the figure, we can see that using \&\& makes the two sub-lemmas a little simpler to prove, however
for the total function lemma, it requires substantially more steps. On the other hand \texttt{andb} requires a bit more steps for the two sub-lemmas but is more than 4 times more efficient for the total function lemma.
Most bizzarely is it that using \texttt{andb} in \texttt{eval\_bexpr} and using \&\& in \texttt{eval\_bexpr\_total\_fun} cannot even be shown, despite the fact that their semantics evidently should be the same.
\begin{table}
  \centering
  \begin{tabular}{c || c | c}
     predicate \textbackslash lemma & \&\& & andb \\
    \hline
    \hline
    \&\& & \begin{array}{r} 12008 \\ 1058 \\ 51046 \end{array} & \begin{array}{r} 56390 \\1058 \\ 51046 \end{array} \\
    \hline
    andb & \begin{array}{r} timeout \\ 1144 \\ 12146 \end{array} & \begin{array}{r} 14508 \\ 1144 \\ 12146 \end{array} \\
  \end{tabular}
  \caption{Table of the number of steps taken to prove the three properties.}
  \label{tab:stepsbexpr}
\end{table}

% && vs andb
% band

\subsection{Statements}
Finally we present our implementation of semantics of statements.
\subsubsection{Semantics}
Once again, we define the semantics by an inductive predicate. The semantics are generally not that complex,
however there are many more cases to account for when dealing with statements.
An example of this is the inference rules for while-statements.
We must consider 4 different cases:

\begin{enumerate}
  \item The boolean condition is true, and thus the body evaluates to some new state, which is used for the next iteration of the while loop.
  \item The boolean condition is false, and thus the loop ends in the same state.
  \item The boolean condition results in abnormal behaviour and thus the entire statement should result in abnormal behaviour.
  \item The body results in abnormal behaviour and likewise the entire loop results in abnormal behaviour.
\end{enumerate}

However for assertions, we only consider a single case, since, as mentioned when presenting the inference rules in \ref{sec:stmtsemantics},
we do not consider assertions in the operational semantics, but rather include them for strengthening the
verification condition of a program.

\subsubsection{Properties}
The properties for asserting the correctness of the implementation of the semantics is built
upon the lemmas regarding totality and determinism of boolean and arithmetic expressions.
Hence we consider the lemma
\texttt{eval_stmt_deterministic}. Unfortunately, our trick from the previous grammar constructs of utilising a recursive lemma does not render useful for statements.
Running the recursive lemma for 300 seconds using Alt-Ergo, does not provide a proof.
There can be multiple reasons for this.

First and foremost we cannot include while-statements in the recursive lemma. The reason for this, is that we have to
include the variant for a recursive lemma to even discharge the proof.
I.e. if we cannot prove termination of a lemma it cannot hold true in the system of why3, since this would only entail partial correctness.
It should be clear from the ESWhileT case of the inference rules, that we do not reduce the structure of s and hence s does not respect the well founded order.
Had we included a variant in WHILE we might have been able to express the termination.

We then considered the proof without including the While in the language. Also in this case we could not find a proof. We then tried to remove If. Yet again, the SMT solvers timed out. Finally we found that removing Sequences, would allow us to prove the determinism.

specifically, when excluding Seq, If, While, it took Alt-Ergo 5956 steps to prove the lemma.
While only excluding Seq and While the prover to 13965 steps. Again this shows how 3 additional inference rules
can increase the number of step by a significant amount.

One perculiarity we further found was that one can compare mappings by = without importing the module map.MapExt (in the standard library), which specifically defines extensionality of mappings.
But again this might be overlapping instances of the build in functionality and the standard library.
However this, along with the conumdrum of \&\& and \texttt{andb}, begs the question as to how well the
standard library is structured.



\subsection{Assertion Language (Formulas)}\label{sec:iformulas}
\paragraph{Formalization.}
As mentioned in \autoref{TODO}, we cannot formalize the assertion language in a similar manner to
the onther language constructs because of quantifers.
We instead decided to use a predicate to describe the semantics.
As Formulas are a logic construct the choice naturally was to use a predicate over a function,
since predicates are part of why3's logical languages and therefore is useful in reasoning about logics.
The predicate \texttt{eval\_closed\_formula} is shown in \autoref{lst:why3formula}.
The semantics for conjuction implication and negation are rather simple and directly follows the corresponding
semantics of why3's logic.
Universal quantification uses the \texttt{forall} defined in why3, and updates the state accordingly.
For the term expression we use the semantics defined earlier for binary expressions.
We state that for a term to be true, the term should evaluate to true under the judgement of binary expressions.
If it evaluates to true, the term is trivially true.
If it does not evaluate to true, there are two potential evaluations.
Either it can evaluate to false or it results in $\bot$.
This is problem in formulas such as $\neg (x \le 10)$ and x is unbound.
The inner expression $x \le 10$ will be a false term and by negation the entire formula is true.
This should not be possible as formulas with unbound variables should not make sense.
Hence the semantics only works for closed formulas.
We therefore need a wrapper function, so we only consider closed form formulas.
The wrapper \texttt{eval\_formula} will first check if any free variables exists
if this is the case, then the formula is false.
In case we have a closed formula, we evalaute the formula using the defined predicate.
Determining if any free variables exists in the formula, we recursively traverse the formula to ensure all variables are bound in the given store,
and in case of universal quantification, we bind the variable to 0,
so it is present in the store and the value is irrelevant.

\begin{lstlisting}[caption={Predicate defining the semantics of formulas},label={lst:why3formula},language=sml]
predicate eval_closed_formula (f: formula 'a) (st: store 'a)  =
  match f with
  | Fterm b -> eval_bexpr b st (Enormal true)
  | Fand f1 f2 -> eval_closed_formula f1 st /\ eval_closed_formula f2 st
  | Fnot f -> not (eval_closed_formula f st)
  | Fimp f1 f2 -> eval_closed_formula f1 st -> eval_closed_formula f2 st
  | Fall y f -> forall n. eval_closed_formula f st[y <- n]
  end

predicate eval_formula (f : formula 'a) (st : store 'a) =
  if is_closed_formula f st then eval_closed_formula f st
  else false
\end{lstlisting}

\paragraph{Why the semantics are non-blocking.}
We mentioned briefly in \autoref{TODO} for semantics of statements that our semantics are non-blocking.
The reason we opted for this, was that we preemptively knew that dealing with quantifiers would be problematic.
We want to make the code extractable to ocaml, and this means we cannot rely on the runtime of why3 and hence
we would need to implement some algorithm to deal with quantified proofs.
This was not an objective of this project.

Possible solutions for implementing a blocking semantics are to either have two different assertion languages,
one for executable assertions (I.e. user-specified assertions for runtime) and another language for
assertions used in the verification condition generation to strengthening the precondition.
In the executable assertions we not allow quantifiers, but all other logical constructors would be included.
In the assertions for VC we include all the constructs.

Another solutions would be to simulate blocking using the predefined constructors.
The expressiveness of binary expressions are equivalent to the assertion language with quantification,
and any assertion which should fail can be simulated using an unbound variable.
For instance, we could have

\begin{lstlisting}[language=haskell]
if assertion then skip else s = x
\end{lstlisting}

where \texttt{assertion} defines the assertion that must hold, and \texttt{x} is an unbound variable.

% - We dont have a good exportable way to deal with quantifiers, a possible solution
% could be to have two different assertion languages, one for blocking and one for reasoning about
% verification conditions.

% bexpr kan udtrykke det samme som formulas uden kvantorer og any blocking can be simulated with an
% unbound variable.


\subsection{Evaluator}\label{sec:evaluator}
With our formalization of the semantics, we can define an evaluator and must do so if we want an extractable evaluator for WHILE.
We showed earlier how, the total function lemmas got folded out and expressed
the total function satisfying the semantics.
The evaluator essentially implements this approach, although modelling the store as a mapping is not satisfactory.
Firstly, the semantics for statements results in a store under normal evaluation
and for an evaluator which returns a mapping to be useful we want the be able to
reason about the final store.
This can be done by mappings, but requires any post processing to have knowledge
of the variables assigned in the program either by the user or computationally.
An easier approach would be to have a data-structure which stores key value pairs.
We ended up using a mutable list of key value pairs, which has some promises about its state.



\subsubsection{Modelling a store}\label{sec:model}
To simulate the store, we had to implement a seperate module \texttt{ImpMap}, presented in \autoref{lst:why3state},
since we ran into a number of roadblocks using the predefined data-structures in the standard library.
These will be explained further in \ref{sec:results}.
\\~\\
We define the store through a record called ``state''.
This record contains two different fields.
Firstly we have \texttt{lst} which is a linked list of a key value pair,
whilst the second field \texttt{model} is a mapping of the same type as the state used for defining the semantics.
Notice here that the model is marked with the ghost keyword. This means that the model can only be used in a logical context
and therefore cannot manipulate any computations.
We use the model for reasoning about the store through the verification conditions.

The module implements three functionalities, \texttt{empty}, \texttt{add} and \texttt{find}.

Instantiation of a state by \texttt{empty} should ensure that the image of the store is \{\texttt{None}\}.

You can add key value pairs to a state using \texttt{add}, which will add the pair to the existing mappings.

Lastly, \texttt{find} tries to find the key in the list and if it does not exist we throw an error.

This use of the model enables us to easily propagate the error throughout the evaluation.
The full implementation can be seen in Appendix \ref{codemodel}.

% \begin{lstlisting}[caption={Model for a store},label={lst:why3state},language=sml]
% type model_t = M.map int (option int)

% predicate match_model (k: int) (v: int) (m : model_t) =
% match M.get m k with
% | None -> false
% | Some v' -> v = v'
% end

% function helper (m : model_t) (pair: (int,int)) : bool =
% let (k,v) = pair in match_model k v m

% type state = { mutable lst : list (int, int);
%                ghost mutable model : model_t
%              }
%       (* invariant { for_all (helper model) lst } *)
%       (* by { lst = Nil ; model = (fun (_ : int) -> None) } *)

% exception Unbound

% function domain (s : state) : M.map int (option int) = s.model

% let function empty () : state
% ensures {forall k. M.get result.model k = None }
% = { lst = Nil ; model = (fun (_ : int) -> None) }

% let add (k: int) (v: int) (s : state) : ()
% writes { s.lst }
% writes { s.model }
% ensures { s.model = M.((old s.model)[k <- Some v]) }
% ensures { hd s.lst = Some (k,v) /\ match_model k v s.model }
% = s.lst <- (Cons (k,v) s.lst);
%   s.model <- M.(s.model[k <- Some v])

% let rec find (k: int) (s : state) : int
%   variant { s.lst }
%   ensures { match_model k result s.model}
%   raises { Unbound -> M.get s.model k = None}
% = match s.lst with
%   | Cons (k', n) s' -> if andb (k <= k') (k' <= k) then n
%      else find k { lst = s'; model = s.model}
%   | Nil -> raise Unbound
%   end
% \end{lstlisting}


\subsubsection{Evaluation}
For the evaluation we define the imperative store, and four functions: one for each construct,
arithmetic expressions, boolean expressions and statements, and a ``top-level'' function which
evaluates a statement and extracts the result of evaluation, and also clearing the store.

In the contract of each function, we describe both what happens under normal execution and what happens under abnormal execution, using the \texttt{ensures} and \texttt{raises}.
For instance we have the evaluator for arithmetic expressions in \autoref{lst:why3aeval}.

\begin{lstlisting}[caption={Evaluation of arithmetic expression},label={lst:why3aeval},language=sml]
let rec aeval (a: aexpr id) : int
  variant { a }
  ensures { eval_aexpr a (IM.domain sigma) (Enormal result) }
  raises { IM.Unbound -> eval_aexpr a (IM.domain sigma) (Eabnormal Eunbound) }
= match a with
  | Acst i -> i
  | Avar v ->  IM.find v sigma
  | Abin a1 op a2 -> (eval_op op) (aeval a1) (aeval a2)
end
\end{lstlisting}

Like for the lemma stating that the semantics
are a total function, we want the evaluator to adhere to the semantics from the inductive predicate.
In the case of an unbound variable the \texttt{Imperative Map find} will raise an exception, and in this case
we will ensure that the behaviour of the evaluation is Eabnormal Eunbound.
It should be clear, that this can easily be extended to other errors, by doing a conjunction of
all possible pairs of exceptions and semantical behaviour.

We follow the same structure for evaluation of boolean expressions and statements.
The verification condition is easily proven for arithmetic and boolean expressions,
but again we cannot show it for statements, neither for total nor partial correctness.


\subsection{Verification Condition Generation}\label{sec:vcg}
The other key part of this project is the formalisation of an extractable verification condition generator.
In \ref{sec:impwlp} we introduce the formalisation of the weakest liberal precondition,
but before we can do so, we need to address the matter of variables in formulas.


\subsubsection{Variable substitution}
From \autoref{fig:wlp} it should be clear that there are multiple situations where we want to substitute
all free occurences of a variable by an new variable.
For instance when $s$ is an assignment $x := a$ in $wlp(s, Q)$ where $Q$ is a formula,
we want to update each occurence of $x$ in $Q$, but do not want to substitute bound instances of $x$.

The rules for substitution of variables are as presented in \autoref{fig:substrules},
where $f_{i}$ is a formula, $x,y,z$ are variables, and $a_{i}$ and $b_{i}$ describes arithmetic and boolean expressions respectively. Conjunction for booleans and formulas are equivalent.

\begin{figure}[h!]
\begin{align*}
\subst{x}{x}{y} &= y \\
\subst{z}{x}{y} &= z \\
&\\
\subst{(a_{1} \leq a_{2})}{x}{y} &= \subst{a_{1}}{x}{y} \le \subst{a_{2}}{x}{y} \\
\subst{(b_{1} \wedge b_{2})}{x}{y} &= \subst{b_{1}}{x}{y} \wedge \subst{b_{2}}{x}{y} \\
\subst{(\neg b)}{x}{y} &= \neg\subst{b}{x}{y} \\
&\\
\subst{(f_{1} \Rightarrow f_{2})}{x}{y} &= \subst{f_{1}}{x}{y} \Rightarrow \subst{f_{2}}{x}{y} \\
\subst{(\forall x. f)}{x}{y} &=  \forall x. f\\
\subst{(\forall z. f)}{x}{y} &=  \forall z. \subst{f}{x}{y}\\
\end{align*}
\caption{Rules for substituting variables in WHILE.}
\label{fig:substrules}
\end{figure}

The substitution must also adhere to the rule that $y$ must be free for $x$ in the expression/formula,
meaning a free occurence of $x$ must not be bound when substituted by $y$.

These rules can now we used for defining a substitution function.
From the formulation of the substitution function it should be easy to see that we recurse down the syntax tree, substituting recursively on the subterms.
The most interesting case is when we meet a quantifier. If the variable $x$ we want to substitute is bound then we stop recursion, as all occurences of $x$ will then be bound.
Otherwise we just continue the recursion.

It should be noted that we do not ensure that $x$ must be free for $y$,
thus the function is rather unsafe.
We only ever use it in a context where this cannot happen, since we instantiate a new variable that does not occur in the formula, bound or free. The variable is so-called fresh in the formula.

The way we generate a fresh variable, is by traversing the syntax tree of the formula and recursively taking the maximal variable value and adding 1.
This works since we use integers as identifiers.
A similar result can be obtained by all infinite enumerable sets. For instance if identifiers were strings, one could add a new character to the longest variable name.

This seems rather trivial, however we formulated predicates that state whether a variable is fresh in an expression or formula. We then afterward tried making a lemma which stated:

\begin{lstlisting}
lemma fresh_var_is_fresh : forall f v.
    v = fresh_from f -> fresh_in_formula v f
\end{lstlisting}

The lemma simply asserts that generating a fresh variable $v$ from $f$ implies that $v$ is fresh in $f$.
We got the inspiration for this approach from \cite{wp-revisited}.
In their work they formulate the \texttt{fresh\_from} function using axioms.
That is they provide a function declaration and then the function ``computes'' the value based on a set of axioms.
For us this is not a viable approach, as such functions cannot be extracted.
Therefore we defined the function ourselves and stated the axiom as a lemma to ensure that \texttt{fresh\_from} always ensures a fresh variable.

Although this seems like a trivial lemma, we were not able to prove this using the SMT solvers.
We tried to define this system in Isabelle to see how difficult it was to prove using a proof assistant, where one has more control, and we succeeded in proving this.

However, just because this is provable in Isabelle it does not mean that it is also possible
for us to do in Why3, as there might be a bug in our code.
Also, it would probably require that the proof was discharged to an automated proof assistant, in which one could tranform the goal to find a proof.

The isabelle proof might not be the cleverest way to prove it, but we proved it the following way.
First let $V$ be the set of all vars in an arithmetic expression $a$.
Then assuming $v$ is the maximum identifier in $a$, we show that $\forall x \in V. v \geq x$. We do so by induction.
By this we can then show $v + 1 \notin V$.
We then made a lemma stating that given $v \notin V$ then $v$ is fresh in $a$.
From this we can show that $v + 1$ is fresh in $V$.
We then do the same for boolean expression and formulas, and the proof is done.
The full proof can be seen in Appendix \ref{codeisabelle}.


\subsubsection{Weakest Liberal Precondition}\label{sec:impwlp}
We now go over the implementation of the weakest liberal precondition calculus, and the different properties that we want to hold for the verification condition generation.

\paragraph{Implementation of Weakest Liberal Precondition calculus.}
We implemented the rules for weakest liberal precondition by a recursive function that directly
follows the inference rules.
We focus on two of the rules, namely assignments and while statements, which both does substitution in the formula.
For assignments we define the rule as follows:

\begin{lstlisting}
| Sass x e -> let y = fresh_from q in
              Fall y (Fimp (Fand (Fterm (Bleq (Avar y) e))
                                 (Fterm (Bleq e (Avar y)))) (subst_fmla q x y))
\end{lstlisting}

We first generate a fresh variable, which is then substituted into the postcondition $Q$.
We further follow the semantics, however since we do not have equality we use $y \leq e \wedge e \leq y$.

The other interesting rule, for while, is implemented as follows:
\begin{lstlisting}
| Swhile cond inv body ->
  Fand inv
       (abstract_effects body
       (Fand (Fimp (Fand (Fterm cond) inv) (wp body inv))
             (Fimp (Fand (Fnot (Fterm cond)) inv) q))
       )
\end{lstlisting}

Here the interesting thing is the function \texttt{abstract\_effects}. This function quantifies over all assigned variables in the body of the loop and substitute the quantifiers with the free variables in the formula
$((cond \wedge inv) \Rightarrow wp(body,inv)) \wedge ((\neg cond \wedge inv) \Rightarrow Q)$.
We need this because the $wlp$ rules dictates that all assigned variables in $s$ should be
substituted with fresh variables, and this function helps us do that.


Like \texttt{fresh\_from} this function is inspired by \cite{wp-revisited}.
Again they use axioms to define the function, whereas we have to actually implement it.
\texttt{abstract\_effects} takes a statement $s$ and formula $f$, and updates the formula if the statement is an assignment.
This is done by creating a fresh variable and substituting it into $f$, and then quantify the freshly made variable.
This is not the most efficient solution, as multiple assignment to the same variable will create unused quantifiers.
All the cases will traverse the abstract syntaxt tree, or end in a leaf and do nothing.
The implementation of \texttt{abstract\_effects} is shown in \autoref{lst:abstracteff}.

\begin{lstlisting}[caption={Implementation of the \texttt{abstract\_effects} function},label={lst:abstracteff},language=sml]
let rec function abstract_effects (s : stmt int) (f : formula int) : formula int
  variant { s }
= match s with
  | Sskip | Sassert _ -> f
  | Sseq s1 s2 | Sif _ s1 s2 -> abstract_effects s2 (abstract_effects s1 f)
  | Sass x _ -> let v = fresh_from f in
                let f' = subst_fmla f x v in
                Fall v f'
  | Swhile _ _ s -> abstract_effects s f
  end
\end{lstlisting}

Whilst we implemented this function directly, we still want the properties for the function to hold.
The following paragraphs describes the different properties. We were not able to prove any of these automatically, for further details see \ref{sec:results}.

\paragraph{Specialization.} Firstly, we have

\begin{lstlisting}
  lemma abstract_effects_specialize : forall st : store int, s f.
    eval_formula (abstract_effects s f) st -> eval_formula f st
\end{lstlisting}

\textbf{TODOOOO}
which states that evaluation of applying abstract effects off s on f in state st implies that f evaluates to true in st. Essentially it states that if we quantify the variables in f and the formula is true under the quantification, then $f$ also hold if the variables are not quantified.
THIS DOES NOT MAKE ANY SENSE??? PLEASE LOOK AT IT TOMORROW.
%%% NOTES %%%
abstract effects property about specialize seems weird, but as the function is defined
from these axioms, it simply states that it must hold for the function, ie. the formula f
must be closed, or else the axiom wouldn't hold.
We can fix this in our implementation by havin a \texttt{requires} that says the formula
must be closed.
Technically we could include all these axioms in an \texttt{ensures}, as they should hold
for the function???? anyways, the way they define the function is wack, but pretty cool.
But obviously we had to do it differently, as we want to be able to extract code.

\paragraph{Quantification over conjunction.}
Secondly, we have
\begin{lstlisting}
lemma abstract_effects_distrib_conj : forall s p q st.
   eval_formula (abstract_effects s p) st /\ eval_formula (abstract_effects s q) st ->
   eval_formula (abstract_effects s (Fand p q)) st
 \end{lstlisting}

Which states that if we apply abstract effects of $s$ on two formulas $p$ and $q$ and they both evaluate to true,
then evaluating the result of applying the abstract effects of $s$ over the conjunction of $p$ and $q$, must also be true.

\paragraph{Monotonicity.}
Thirdly, we have the property of monotonicity.

\begin{lstlisting}
lemma abstract_effects_monotonic : forall s p q.
   valid_formula (Fimp p q) ->
   forall st. eval_formula (abstract_effects s p) st ->
   eval_formula (abstract_effects s q) st
 \end{lstlisting}

Essentially what monotonicity states, is that applying additional assumptions to a formula will not change the meaning. It should be noted that we want to quantify all states in the entailment, as the property does not hold for a fixed state.

\paragraph{Invariance.}
Lastly, we consider the notion of invariance.
In \cite{wp-revisited} the axiomatized version of the following lemma, is used to define which variables should be quantified.
We explicitly state this in the body of \texttt{abstract\_effects},
but for good measure we include it as a lemma.

\begin{lstlisting}
lemma abstract_effect_writes : forall st s q.
   eval_formula (abstract_effects s q) st ->
   eval_formula (wp s (abstract_effects s q)) st
 \end{lstlisting}

 The property states that if a the formula $q$ abstracted by $s$ is true in some state,
 then the weakest precondition on the same abstraction on $q$ by $s$ should also be true.
% When \cite{} uses this as an axiom, it ensures, that \texttt{abstract\_effects}

% \paragraph{Proofs of properties for abstract\_effects}
% We have not been able to automatically prove these lemmas in why3.
% At the moment we have a partial proof for the function in Isabelle.
% ???? WRITE SOME MORE ????

\paragraph{Properties and Soundness of WLP.}
One of the main goals in making a formally verified Verification condition generator is to ensure the correctness of the implementation.
We consider the correctness through its soundness.
To do so we must consider two of the same properties we just stated for \texttt{abstract\_effects}, namely monotonicity and conjunction distribution.

Monotonicity of WLP means that if for two formulas $p$ and $q$ the formula $\vDash p \Rightarrow q$ holds, then $\vDash wp(s, p) \Rightarrow wp(s,q)$ also holds.
Notice again that this must hold for all statements and states.

Distribution of weakest precondition over conjuction is similar to the lemma of abstract effects, except that the transformation on $p$ and $q$ are now considered for $wp$.

Both lemmas can be seen in \autoref{lst:lemmaswlp}.

\begin{lstlisting}[caption={Lemmas for stating monotonicity and distribution over conjunction for wlp},label={lst:lemmaswlp},language=sml]
lemma monotonicity: forall s p q.
      valid_formula (Fimp p q) -> valid_formula (Fimp (wp s p) (wp s q))

lemma distrib_conj: forall s sigma p q.
      eval_formula (wp s p) sigma /\ eval_formula (wp s q) sigma ->
      eval_formula (wp s (Fand p q)) sigma
\end{lstlisting}

Once again we not able to directly show these lemmas.
We tried to unfold the recursion on $s$, in a similar manner to how we proved determinism, but
 did not achieve anything by doing this.
We have a formalized proof of both properties in Isabelle.
The proof for monotonicity can be seen in \autoref{lst:isamono}.
As mentioned we prove the lemma by induction on $s$,
and mark $p$ and $q$ as arbitrary, since the lemma should hold for any non-fixed $p$ and $q$.
The proof only shows the cases for sequences, assignments and while.
The case for sequences are actually directly proved from the assumptions.
We only distinguish this case because we need to simplify the other trivial cases.
For both assignment and while we used the sledgehammer to find the proofs by metis,
which is a complete automatic theorem prover for first order logic with equality\cite{sledgehammer}.

\begin{lstlisting}[caption={Proof of monotonicity in Isabelle},label={lst:isamono},language=sml]
lemma monotonicity : "valid_formula (FImp p q) \<Longrightarrow>
  valid_formula (FImp (wp s p) (wp s q))"
proof(induction s arbitrary: p q)
  case (SSeq s1 s2)
  then show ?case by auto
next
  case (Sassign x1 x2)
  then show ?case
    by (metis abstract_effect_writes abstract_effects.simps(2)
         eval_formula.simps(2) valid_formula_def wp.simps(3))
next
  case (Swhile x1 x2 s)
  then show ?case
    by (metis abstract_effect_writes abstract_effects.simps(2)
        eval_formula.simps(2) valid_formula_def wp.simps(3))
qed (simp_all add: valid_formula_def)
\end{lstlisting}

For distribution over a conjuction, the proof is straight forward.
We do induction with the same setup.
Again the sequences, assignments and while cannot trivially be proved by simplification,
but the same metis proof used for monotonicity can be used for all three cases.

With these two properties, we should be able to show soundness for the wp function.
We decided to not prove the completeness of the function because this says something about the expressiveness of the function, whereas it is more important to ensure that
the function is correct.
Using soundness to show correctness is two-fold.
On the one hand showing the soundness of $wp$ ensures that $\hoare{wp(s,Q)}{s}{Q}$ is valid for partial correctness for all $s$ and $q$, ensures that we cannot generate invalid verification conditions.
On the other hand, we already know weakest precondition to be sound and proving it for the function $wp$ ensures that our implementation adheres to the semantics or atleast an equivalent semantic.
We can formalize the soundness proof as follows:

\begin{lstlisting}[caption={WLP soundness lemma in Why3},label={lst:isamono},language=sml]
  lemma soundness_wp : forall st st' : store int, s, q.
  eval_stmt s st (Enormal st') /\ eval_formula (wp s q) st -> eval_formula q st'
\end{lstlisting}

The lemma states that if $s$ terminates normally in $st$ and $wp(s,q)$ then $q$ must also hold.
This is exactly, what we formally proved by pen-and-paper in \ref{sec:wp}.
We have not been able to prove this. But obviously we would not be able to show this automatically since it requires induction.



\section{Extraction of code}\label{sec:extract}
Why3 makes it possible to extract whyML code to either Ocaml or C code.
One of the main goals of this project was to make the verified code extractable.
To achieve this we had to adhere to some limitations.
Mostly these are related the expressiveness and hierachy of whyML.

\paragraph{Extraction is correct by construction.}
The extraction of whyML programs is correct by construction.
This means that each syntactical object is directly translated into an equivalent object
in the target language. For instance a program function written in whyML denoted by either \texttt{let}, \texttt{let function}, \texttt{let rec function} or \texttt{let rec} directly translate into its equivalent Ocaml function. This ensures that extracted code does the same when exported.
On the downside this means that we can only extract our code to Ocaml, but not to C, since there is no
language defined notion of Abstract Data Types.
Thus our choice to represent our object language as ADT's have limited our ability to extract code.
However structuring the code in C friendly manner would likely become quite tedious.

\paragraph{Program functions and Logical functions.}
Just as the object level limits what we can do in terms of extracted code, so is the logical level.
There is a clear distinction between logical functions and program functions in why3.
All program functions is specified with a \texttt{let} and can be extracted,
while the logical functions can be used to reason about the program functions.
Actually extracting the code has been a bit of a challenge.
Firstly, as mentioned earlier we compromised in regards to assertions in the operational semantics of the program. For the reason that it is hard to argue about logical quantifiers on a program level, while it is much easier to do on a logical level. Hence why evaluation of Formulas is done with a predicate and not a program function.
Furthermore, making the actual evaluator was quite troublesome.
For defining the actual semantics we used inductive predicates, which is a logical construct, and this is extremely useful because we can argue the correctness of evaluation of a statement on a program level.
In this reasoning we used maps, which is simply functions, with some syntactical sugar for updates and computations.
The problem arises when we need to define the environment for the actual interpreter.

We were not able to find any module which was extractable and would adhere to the same logical meaning as that of maps.
Hence we had to implement the module explained in \ref{sec:env}.
And while there might be better ways to handle the store, Our current approach seems to suffice.

In regards to the verification condition generation, we did not run into too much trouble.
As mentioned previously our approach to the weakest precondition generation follows the same structure as WP revisited in why3\cite{},
but deviates in how functions are defined.
Axiomatized functions can clearly not be exported,
since they dont have a function body but simply must adhere to a set of axioms.
And therfore we need to prove explicitly that our function is correct,
whereas axiomatization will be correct by definition.
But actually extracting the $wp$ function is rather simple, as it essentially just transforms an ADT.
And clearly can be defined in on the program level.


\paragraph{Using the extracted code.}
The code is essentially split into two different functionalities.
The evaluator does imperative evaluation on some statement in the mutable environment.
We can extract the code for evaluation by the following command:

\begin{lstlisting}
why3 extract --recursive -D ocaml64 -L . evaluator.mlw -o eval.ml
\end{lstlisting}

This will tell to recursively add dependencies into the module defined in the evaluator.mlw file, using the ocaml64 driver and create a module in the file eval.ml.
We can then use the module in an ocaml project (where we might define a parser etc. for the language).
\autoref{lst:ocamlexample} shows an example of a very simple program.
The program, defines two WHILE programs, which are very simple.
\texttt{prog} is a simple assignment. One thing to note is that because we used the int type in why3, we have to use the \(Z.of\_int\) because int's in why3 are unbounded integers from the zarith library.
\texttt{prog'} defines a slightly more complicated program, which does 2 assignments, and the second assignment tries to assign an expression with an unbound variable to a variable.

\begin{lstlisting}[caption={ocaml program using the evaluator},label={lst:ocamlexample},language=sml]
open Eval
let prog =
             Sass  (Z.of_int 1, ABin (
                        Acst (Z.of_int 5)
                        , Mul
                        , Acst (Z.of_int 10)))
let prog' = Sseq (
             Sass  (Z.of_int 1, ABin (
                        Acst (Z.of_int 5)
                        , Mul
                        , Acst (Z.of_int 10)))
             , Sass (Z.of_int 2 , ABin (
                         Acst (Z.of_int 5)
                       , Add
                       , Avar (Z.of_int 3))))
let res = eval_prog prog
let () = List.iter (fun (k,v) -> Format.printf "%d, %d\n" (Z.to_int k) (Z.to_int v)) res
let () = List.iter (fun (k,v) -> Format.printf "%d, %d\n" (Z.to_int k) (Z.to_int v)) sigma.lst
let res' = eval_prog prog'
\end{lstlisting}

Compiling the program with

\begin{lstlisting}
\textdollar ocamlbuild -pkg zarith main.native
\end{lstlisting}

and running it by:

\begin{lstlisting}
\textdollar ./main.native
\end{lstlisting}

The following output is produced:

\begin{lstlisting}
1, 50
Fatal error: exception Eval.Unbound
\end{lstlisting}

The evaluation of the first program is that variable 1 is 50. The second List.iter prints nothing,
this shows that we correctly resets the environment. Lastly \texttt{prog'} correctly produces an Unbound exception.
\\~\\
The procedure for the verification condition generator is the same.
It is worth noting that we have not implemented any way to use the generated formula for anything.
And thus it may render useless for now to extract it.
In case one want to actually verify the correctness of a WHILE program, we can use the povers of Why3 and use the meta level of logic using a goal.
Take the following program

\begin{lstlisting}[caption={WHILE program which multiples q and r by repeated addition},label={lst:whileexample},language=sml]
q = 10;
r = 5;
res = 0;
ghost = q;
while (q > 0)
invariant {(res $\leq$ (ghost - q) * r && (ghost - q) * r $\leq$ res) /\ 0 \leq q} {
      res = res + r;
      q = q - 1;
};
assert {res $\leq$ ghost * r /\ ghost * r $\leq$ res};
\end{lstlisting}

The syntax should be straight forward. The program will define 4 variables, $q,r,res$ and $ghost$.
The program will execute the while loop $q$ times. In the body of the loop we add $r$ to $res$ and decrement $q$ by 1. We keep as invariant that $res = (ghost - q) * r$. Lastly we assert that $res$ is the orignal value of $q$, described by $ghost$, multiplied with $r$.
Writing this as an AST, we have:

\begin{lstlisting}[caption={WHILE AST in why3},label={lst:whileast},language=sml]
let function prog () =
  (* var 2 is q *)
  (Sseq (Sass 2 (Acst 10))
  (* var 1 is res *)
  (Sseq (Sass 1 (Acst 0))
  (* var 0 is r *)
  (Sseq (Sass 0 (Acst 5))
  (* Ghost var = q *)
  (Sseq (Sass (-1) (Avar 2))

  (Sseq (Swhile (Bleq (Acst 0) (ABin (Avar 2) Add (Acst 1)))
        (* res <= (ghost - q) * r /\ res >= (ghost - q) * r *)
        (Fand (Fterm (Band (Bleq (Avar 1)  (ABin (ABin (Avar (-1)) Sub (Avar 2)) Mul (Avar 0)))
             (Bleq (ABin (ABin (Avar (-1)) Sub (Avar 2)) Mul (Avar 0)) (Avar 1))))
        (* 0 <= q *)
             (Fterm (Bleq (Acst 0) (Avar 2))))
        (* NOW BODY OF THE LOOP *)
        (Sseq (Sass 1 (ABin (Avar 1) Add (Avar 0)))
              (Sass 2 (ABin (Avar 2) Sub (Acst 1)))))

        (* Assert {res = ghost * r} *)
        (Sassert (Fterm (Band (Bleq (Avar 1) (ABin (Avar (-1)) Mul (Avar 0)))
        (Bleq (ABin (Avar (-1)) Mul (Avar 0)) (Avar 1))))))))))
      \end{lstlisting}

We can then construct a goal checking if the weakest precondition of this program is valid.

\begin{lstlisting}
goal mult_is_correct :
   valid_formula (wp (prog ()) (Fterm (BCst true)))
\end{lstlisting}

Feeding this to an SMT solver such as Eprover, we can show the program to be correct, in around half a second.

