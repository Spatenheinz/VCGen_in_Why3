In this section we will describe the formalization of formulas, ie. the assertion language of WHILE.

\paragraph{Formalization.}
As mentioned in \ref{sec:assert}, we cannot formalize the assertion language in a similar manner to
the other language constructs because of quantifers.
We instead decided to use a predicate to describe the semantics.
As formulas are logic constructs the choice naturally was to use a predicate over a function,
since predicates are part of why3's logical languages and therefore is useful in reasoning about logics.

We want to formalize that the evaluation of a formula depends on the evaluation of subterms
of that formula. This will eventually use the evaluation of boolean expressions, and thus
the result will be whether the formula holds or not.
This is formalized in the predicate \texttt{eval\_closed\_formula}, shown in \autoref{lst:why3formula}.
\\~\\
The semantics for conjuction, implication and negation are rather simple and directly follows the corresponding
semantics of why3's logic.
Universal quantification uses the \texttt{forall} defined in why3, and updates the state accordingly.

For the term expression we use the semantics defined earlier for binary expressions.
We state that for a term to be true, the term should evaluate to true under the judgement of binary expressions.
If it evaluates to true, the term is trivially true.
If it does not evaluate to true, there are two potential evaluations.
Either it can evaluate to false or it results in $\bot$.
This poses a problem in formulas such as $\neg (x \leq 10)$ where $x$ is unbound.
The inner expression $x \leq 10$ will be a false term, as $x$ is unbound, and by negation the entire formula is true.
This should not be possible as formulas with unbound variables should not make sense.
Hence the semantics only works for closed formulas, and we therefore need a wrapper function, to ensure that we only consider closed formulas.

The wrapper \texttt{eval\_formula} will first check if any free variables exists, and
if this is the case the formula evaluates to false.
In case we have a closed formula, we evaluate the formula using the defined predicate.
\\~\\
To determine whether any free variables exists in the formula, we define a predicate
\texttt{is\_closed} which takes a formula and a store and recursively traverse the formula to ensure all occurring variables are bound in the given store.
In case of universal quantification, we bind the quantified variable to 0,
so it is present in the store. The value in this case is irrelevant, as we just want to determine whether the formula is closed or not, and not what it evaluates to.
The \texttt{eval\_formula} and \texttt{eval\_closed\_formula} predicates are shown in
\autoref{lst:why3formula}. For the rest of the implementation, see Appendix \ref{codeformulas}.

\begin{lstlisting}[caption={Predicate defining the semantics of formulas},label={lst:why3formula},language=sml]
predicate eval_closed_formula (f: formula 'a) (st: store 'a)  =
  match f with
  | Fterm b -> eval_bexpr b st (Enormal true)
  | Fand f1 f2 -> eval_closed_formula f1 st /\ eval_closed_formula f2 st
  | Fnot f -> not (eval_closed_formula f st)
  | Fimp f1 f2 -> eval_closed_formula f1 st -> eval_closed_formula f2 st
  | Fall y f -> forall n. eval_closed_formula f st[y <- n]
  end

predicate eval_formula (f : formula 'a) (st : store 'a) =
  if is_closed_formula f st then eval_closed_formula f st
  else false
\end{lstlisting}

\paragraph{Blocking and non-blocking.}
We mentioned briefly in \ref{sec:stmtsemantics} that the semantics for formulas in statements
are non-blocking.
The reason we opted for this approach, was that we preemptively knew that dealing with quantifiers could be problematic.
We want to make the code extractable to OCaml, and this means we cannot rely on the runtime of why3 and hence
we would need to implement some algorithm to deal with quantified proofs.
This was not an objective of this project, and thus we chose a non-blocking semantics.

Possible solutions for implementing a blocking semantics are to either have two different assertion languages,
one for executable assertions, i.e. user-specified assertions to be evaluated during runtime,
and another language for
assertions used in the verification condition generation to strengthening the precondition.
In the executable assertions we do not allow quantifiers, but all other logical constructors would be included.
In the assertions for VC we would include all the constructs.

Another solution would be to simulate blocking using the predefined constructors.
This means we would use boolean expressions for assertions in statements, as the expressiveness of boolean expressions are equivalent to the assertion language without quantification,
and any assertion which should fail can be simulated using an unbound variable.
For instance, we could formalize the semantic rules for assertions in statements as follows:

\begin{lstlisting}[language=sml]
| ESAssert     : forall f, st : store 'a.
                 eval_bexpr f st (Enormal true) ->
                 eval_stmt (Sassert f) st (Enormal st)
| ESAssert_err1: forall f, st : store 'a.
                 eval_bexpr f st (Enormal false) ->
                 eval_stmt (Sassert f) st (Eabnormal Eassert)
| ESAssert_err2: forall f, st : store 'a.
                 eval_bexpr f st (Eabnormal err) ->
                 eval_stmt (Sassert f) st (Eabnormal err)
\end{lstlisting}
where \texttt{Eassert} is a new error for failing assertions, that we would need to add to
the set of valid errors.

In the evaluator this could be implemented as follows:
\begin{lstlisting}[language=haskell]
| Sassert b -> if b then () else aeval x; ()
\end{lstlisting}
where the variable $x$ is unbound, thus provoking an unbound error, ensuring that the program
cannot be proven correct.


% - We dont have a good exportable way to deal with quantifiers, a possible solution
% could be to have two different assertion languages, one for blocking and one for reasoning about
% verification conditions.

% bexpr kan udtrykke det samme som formulas uden kvantorer og any blocking can be simulated with an
% unbound variable.
