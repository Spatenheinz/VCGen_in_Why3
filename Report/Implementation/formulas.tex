\paragraph{Formalization.}
As mentioned in \autoref{TODO}, we cannot formalize the assertion language in a similar manner to
the onther language constructs because of quantifers.
We instead decided to use a predicate to describe the semantics.
As Formulas are a logic construct the choice naturally was to use a predicate over a function,
since predicates are part of why3's logical languages and therefore is useful in reasoning about logics.
The predicate \texttt{eval\_closed\_formula} is shown in \autoref{lst:why3formula}.
The semantics for conjuction implication and negation are rather simple and directly follows the corresponding
semantics of why3's logic.
Universal quantification uses the \texttt{forall} defined in why3, and updates the state accordingly.
For the term expression we use the semantics defined earlier for binary expressions.
We state that for a term to be true, the term should evaluate to true under the judgement of binary expressions.
If it evaluates to true, the term is trivially true.
If it does not evaluate to true, there are two potential evaluations.
Either it can evaluate to false or it results in $\bot$.
This is problem in formulas such as $\neg (x \le 10)$ and x is unbound.
The inner expression $x \le 10$ will be a false term and by negation the entire formula is true.
This should not be possible as formulas with unbound variables should not make sense.
Hence the semantics only works for closed formulas.
We therefore need a wrapper function, so we only consider closed form formulas.
The wrapper \texttt{eval\_formula} will first check if any free variables exists
if this is the case, then the formula is false.
In case we have a closed formula, we evalaute the formula using the defined predicate.
Determining if any free variables exists in the formula, we recursively traverse the formula to ensure all variables are bound in the given store,
and in case of universal quantification, we bind the variable to 0,
so it is present in the store and the value is irrelevant.

\begin{lstlisting}[caption={Predicate defining the semantics of formulas},label={lst:why3formula},language=sml]
predicate eval_closed_formula (f: formula 'a) (st: store 'a)  =
  match f with
  | Fterm b -> eval_bexpr b st (Enormal true)
  | Fand f1 f2 -> eval_closed_formula f1 st /\ eval_closed_formula f2 st
  | Fnot f -> not (eval_closed_formula f st)
  | Fimp f1 f2 -> eval_closed_formula f1 st -> eval_closed_formula f2 st
  | Fall y f -> forall n. eval_closed_formula f st[y <- n]
  end

predicate eval_formula (f : formula 'a) (st : store 'a) =
  if is_closed_formula f st then eval_closed_formula f st
  else false
\end{lstlisting}

\paragraph{Why the semantics are non-blocking.}
We mentioned briefly in \autoref{TODO} for semantics of statements that our semantics are non-blocking.
The reason we opted for this, was that we preemptively knew that dealing with quantifiers would be problematic.
We want to make the code extractable to ocaml, and this means we cannot rely on the runtime of why3 and hence
we would need to implement some algorithm to deal with quantified proofs.
This was not an objective of this project.

Possible solutions for implementing a blocking semantics are to either have two different assertion languages,
one for executable assertions (I.e. user-specified assertions for runtime) and another language for
assertions used in the verification condition generation to strengthening the precondition.
In the executable assertions we not allow quantifiers, but all other logical constructors would be included.
In the assertions for VC we include all the constructs.

Another solutions would be to simulate blocking using the predefined constructors.
The expressiveness of binary expressions are equivalent to the assertion language with quantification,
and any assertion which should fail can be simulated using an unbound variable.
For instance, we could have

\begin{lstlisting}[language=haskell]
if assertion then skip else s = x
\end{lstlisting}

where \texttt{assertion} defines the assertion that must hold, and \texttt{x} is an unbound variable.

% - We dont have a good exportable way to deal with quantifiers, a possible solution
% could be to have two different assertion languages, one for blocking and one for reasoning about
% verification conditions.

% bexpr kan udtrykke det samme som formulas uden kvantorer og any blocking can be simulated with an
% unbound variable.
