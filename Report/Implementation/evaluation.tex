For the evaluation we define the imperative store, and four functions, one for each construct,
arithmetic epressions, boolean expressions and statements, and a ``top-level'' function which evaluates
evaluates a statement and extracts the result of evaluation, and also clearing the store.

In the contract of each function, we describe both what happens under normal execution and what happens under abnormal execution, using the \texttt{ensures} and \texttt{raises}, for instance we have:

\begin{lstlisting}[caption={Evaluation of arithmetic expression},label={lst:why3aeval},language=sml]
let rec aeval (a: aexpr id) : int
  variant { a }
  ensures { eval_aexpr a (IM.domain sigma) (Enormal result) }
  raises { IM.Unbound -> eval_aexpr a (IM.domain sigma) (Eabnormal Eunbound) }
= match a with
  | Acst i -> i
  | Avar v ->  IM.find v sigma
  | ABin a1 op a2 -> (eval_op op) (aeval a1) (aeval a2)
end
\end{lstlisting}

\autoref{lst:why3aeval} present the evaluation of arithmetic expressions. Like for the lemma about the semantics
being a total function, we want the evaluator to adhere to the semantics from the inductive predicate.
In the case of an unbound variable the \texttt{Imperative Map find} will raise an exception, in this case
we will ensure that the behaviour of the evaluation is Eabnormal Eunbound.
It should be clear, that this can easily be extended to other errors, by doing a conjunction of
all possible (exception $\rightarrow$ semantical behaviour) pairs.
We follow the same structure for evaluation of boolean expressions and statements.
The vc is easily proved for arithmetic and boolean expressions.
Again we cannot show it for statements, neither for total nor partial correctness.