Firstly our store $\sigma$ is defined as a mapping from a polymorphic type to an Option Int.
We define the syntax by simple ADTs, which closely resembles the syntax described earlier.

\begin{lstlisting}[caption={Syntax if the WHILE language, implemented in Why3},label={lst:why3syntax},language=sml]
type store 'a = 'a -> option int

type aop = Mul | Sub

type aexpr 'a =
     | Acst int
     | Avar 'a
     | Abin (aexpr 'a) aop (aexpr 'a)

type bexpr 'a =
     | Bcst
     | Bleq (aexpr 'a) (aexpr 'a)
     | Band (bexpr 'a) (bexpr 'a)
     | Bnot (bexpr 'a)

type formula 'a =
     | Fterm (bexpr 'a)
     | Fand (formula 'a) (formula 'a)
     | Fnot (formula 'a)
     | Fimp (formula 'a) (formula 'a)
     | Fall 'a (formula 'a)

type stmt 'a =
     | Sskip
     | Sass 'a (aexpr 'a)
     | Sseq (stmt 'a) (stmt 'a)
     | Sassert (formula 'a)
     | Sif (bexpr 'a) (stmt 'a) (stmt 'a)
     | Swhile (bexpr 'a) (formula 'a) (stmt 'a)

type prog 'a = stmt 'a
\end{lstlisting}

\autoref{lst:why3syntax} shows the implementation of the syntax.
Each syntactical construct has its own type. The types are declared in a straighforward manner.
For example, a boolean expression $a_{0} \land a_{1}$ written as
\[ \texttt{Band a0 a1} \]
in this abstract syntax, and a while-statement $\mathbf{while} \; b \; \mathbf{invariant} \; f \; \{s\}$ is written as
\[ \texttt{SWhile b f s} \]

Now we move on to the semantics of the different constructs, which is a bit more tricky than
the syntax. The following subsections presents the imlementation of these semantics.
