\subsubsection{Semantics}
The formulation of boolean expressions follows the same pattern as arithmetic expressions.
We define an inductive predicate stating the inference rules in \autoref{fig:bexprsemantics}.
As this follows the same structure as for arithmetic expressions we exclude the code here,
for reference see \autoref{appendix???}.

\subsubsection{Properties}
Likewise the properties we use for reasoning about correctness of our boolean semantics resemble the properties for arithmetic expressions. Thus we have the following lemmas:

\begin{enumerate}
  \item \texttt{eval\_bexpr\_fun}
  \item \texttt{eval\_bexpr\_total}
  \item \texttt{eval\_bexpr\_total\_fun}
\end{enumerate}

In the formulation of the boolean semantics, we initially encountered some problems in proving the properties.
In the first iteration of the inductive predicate we used the built in operator \&\&, which works for boolean values.
We then became aware of the function \texttt{andb}, which is defined in the standard library in module bool.Bool.
\texttt{andb} is short-circuiting and presumably the \&\& operator is as well.

From the properties it also seem the two functions have equivalent semantics.
Interestingly they do not require the same amount of steps in
the proofs. \autoref{tab:stepsbexpr} shows the result of proving the properties with different combinations of use of the two operators.
The columns define the operator used in the lemma \texttt{eval\_bexpr\_total\_fun},
and the rows define which operator is used in the inductive predicate.
Each line in a cell corresponds to the number of steps required for Alt-Ergo to prove the lemma, stated in the order from above.

From the figure, we can see that using \&\& makes the two sub-lemmas a little simpler to prove, however
for the total function lemma, it requires significantly more steps.
On the other hand \texttt{andb} requires a bit more steps for the two sub-lemmas but is more than 4 times more efficient for the total function lemma.
Most bizzarely is it that using \texttt{andb} in \texttt{eval\_bexpr} and using \&\& in \texttt{eval\_bexpr\_total\_fun} cannot even be shown, despite the fact that their semantics evidently should be the same.
As a result of these considerations we ended up using \texttt{andb} in our implementation.
\begin{table}
  \centering
  \begin{tabular}{c || c | c}
     predicate \textbackslash lemma & \&\& & andb \\
    \hline
    \hline
    \&\& & \begin{array}{r} 12008 \\ 1058 \\ 51046 \end{array} & \begin{array}{r} 56390 \\1058 \\ 51046 \end{array} \\
    \hline
    andb & \begin{array}{r} timeout \\ 1144 \\ 12146 \end{array} & \begin{array}{r} 14508 \\ 1144 \\ 12146 \end{array} \\
  \end{tabular}
  \caption{Table of the number of steps taken to prove the three properties.}
  \label{tab:stepsbexpr}
\end{table}

% && vs andb
% band
