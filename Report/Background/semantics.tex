We describe WHILE by its natural semantics.
We use a store to keep the values of any program variables, and this is implemented using a finite map:
$$\sigma \in \Sigma = Var \longrightarrow \mathbb{Z} \cup \{\bot\}$$
where $Var$ denotes all possible variables and $\bot$ denotes the abnormal result in case of unbound variables.
Hence we consider a store to be a total function.

\paragraph{Arithmetic expressions.}
We define the judgement for arithmetic expressions as
\[
  \judgea{a,\sigma}{(n | \bot)}
\]
where $n \in \mathbb{Z}$, $a$ is an arithmetic expression, and $x | y$ denotes either $x$ or $y$.
All the inference rules for arithmetic expressions are shown in \autoref{fig:aexprsemantics}.
The rules intuitively describes evaluation of expressions. Not that when encountering abnormal behaviour $\bot$ the behaviour is propagated.

\begin{figure}[h!]
\begin{equation*}
\inference[EACst ]{}
{\judgea{n,\sigma}{n}}
\end{equation*}

\begin{equation*}
\inference[EAVar ]{}
{\judgea{v,\sigma}{n}}[$(\sigma(v) = n)$]
\qquad
\inference[EAVar\_Err]{}
{\judgea{v,\sigma}{\bot}}[$( \sigma(v) = \bot )$]
\end{equation*}

\begin{equation*}
\inference[EABin ]{\judgea{a_{0}, \sigma}{n_{0}} & \judgea{a_{1}, \sigma}{n_{1}}}
{\judgea{a_{0} \oplus a_{1},\sigma}{n_{0} \oplus n_{1}}}
\end{equation*}

\begin{equation*}
\inference[EABin\_Err1 ]{\judgea{a_{0}, \sigma}{\bot}}
{\judgea{a_{0} \oplus a_{1},\sigma}{\bot}}
\qquad
\inference[EABin\_Err2 ]{\judgea{a_{0}, \sigma}{n_{0}} & \judgea{a_{1}, \sigma}{\bot}}
{\judgea{a_{0} \oplus a_{1},\sigma}{\bot}}
\end{equation*}
\caption{Semantics of arithmetic expressions in WHILE.}
\label{fig:aexprsemantics}
\end{figure}

\paragraph{Boolean expressions.}
In a similar manner we define a judgement for boolean expressions as
$$\judgeb{b,\sigma}{(t | \bot)}$$
where $t \in \{true, false\}$ and $b$ is a boolean expression.
The semantics for boolean expressions are presented in \autoref{fig:bexprsemantics}.

\begin{figure}[h!]
\begin{equation*}
\inference[EBCst ]{}
{\judgeb{t,\sigma}{t}}
\end{equation*}
\begin{equation*}
\inference[EBLeq ]{\judgeb{a_0,\sigma}{n_0} & \judgeb{a_1,\sigma}{n_1}}
{\judgeb{a_0 \leq a_1 ,\sigma}{(n_0 \leq n_1)}}
\end{equation*}
\begin{equation*}
\inference[EBLeq\_Err1 ]{\judgeb{a_0,\sigma}{\bot}}
{\judgeb{a_0 \leq a_1, \sigma}{\bot}}
\qquad
\inference[EBLeq\_Err2 ]{\judgeb{a_0,\sigma}{n_0} & \judgeb{a_1,\sigma}{\bot}}
{\judgeb{a_0 \leq a_1, \sigma}{\bot}}
\end{equation*}
\begin{equation*}
\inference[EBAnd ]{\judgeb{b_0, \sigma}{b_0'} & \judgeb{b_1, \sigma}{b_1'}}
{\judgeb{b_0 \land b_1, \sigma}{b_0' \land b_1'}}
\end{equation*}
\begin{equation*}
\inference[EBAnd\_Err1 ]{\judgeb{b_0, \sigma}{\bot}}
{\judgeb{b_0 \land b_1, \sigma}{\bot}}
\qquad
\inference[EBAnd\_Err2 ]{\judgeb{b_0, \sigma}{b_0'} & \judgeb{b_1, \sigma}{\bot}}
{\judgeb{b_0 \land b_1, \sigma}{\bot}}
\end{equation*}
\begin{equation*}
\inference[EBNot ]{\judgeb{b,\sigma}{b'}}
{\judgeb{ \neg b, \sigma}{\neg b'}}
\qquad
\inference[EBNot\_Err ]{\judgeb{b,\sigma}{\bot}}
{\judgeb{ \neg b, \sigma}{\bot}}
\end{equation*}
\caption{Semantics of boolean expressions in WHILE.}
\label{fig:bexprsemantics}
\end{figure}

\paragraph{Statements.}\label{sec:stmtsemantics}
Likewise for statements a judgement will either end in a new state, or an abnormal behaviour, caused by unbound variables. Thus the judgement is defined by
$$\judges{s, \sigma}{(\sigma | \bot)}$$
where $s$ is a statement.
The semantics of statements are presented in \autoref{fig:stmtsemantics}.

\begin{figure}[h!]
\begin{equation*}
\inference[ESSkip ]{}
{\judges{\mathbf{skip},\sigma}{\sigma}}
\end{equation*}
\begin{equation*}
\inference[ESAss ]{\judges{a,\sigma}{n}}
{\judges{x \mathbf{:=} a,\sigma}{\sigma[x \mapsto n]}}
\qquad
\inference[ESAss\_Err ]{\judges{a,\sigma}{\bot}}
{\judges{x \mathbf{:=} a,\sigma}{\bot}}
\end{equation*}
\begin{equation*}
\inference[ESSeq ]{\judges{s_0, \sigma}{\sigma''} & \judges{s_1, \sigma''}{\sigma'} \; \; \; \text{where} \; \sigma'' \neq \bot}
{\judges{s_0 \; ; \; s_1, \sigma}{\sigma'}}
\end{equation*}
\begin{equation*}
\inference[ESSeq\_Err ]{\judges{s_0, \sigma}{\bot}}
{\judges{s_0 \; ; \; s_1, \sigma}{\bot}}
\qquad
\inference[ESIfT ]{\judges{b, \sigma}{true} & \judges{s_0, \sigma}{\sigma'}}
{\judges{\mathbf{if} \; b \; \{ \; s_0 \; \} \mathbf{else} \; \{ \; s_1 \; \}, \sigma}{\sigma'}}
\end{equation*}
\begin{equation*}
\inference[ESIfF ]{\judges{b, \sigma}{false} & \judges{s_1, \sigma}{\sigma'}}
{\judges{\mathbf{if} \; b \; \{ \; s_0 \; \} \mathbf{else} \; \{ \; s_1 \; \}, \sigma}{\sigma'}}
\qquad
\inference[ESIf\_Err ]{\judges{b, \sigma}{\bot}}
{\judges{\mathbf{if} \; b \; \{ \; s_0 \; \} \; \mathbf{else} \; \{ \; s_1 \; \}, \sigma}{\bot}}
\end{equation*}
\begin{equation*}
\inference[ESWhileT ]{\judges{b, \sigma}{true} & \judges{s, \sigma}{\sigma''} & \judges{\mathbf{while} \; b \; \mathbf{invariant} \; f \; \{ \; s \; \}, \sigma''}{\sigma'}}
{\judges{\mathbf{while} \; b \; \mathbf{invariant} \; f \; \{ \; s \; \}, \sigma}{\sigma'}}
\end{equation*}
\begin{equation*}
\inference[ESWhileF ]{\judges{b, \sigma}{false}}
{\judges{\mathbf{while} \; b \; \mathbf{invariant} \; f \; \{ \; s \; \}, \sigma}{\sigma}}
\end{equation*}
\begin{equation*}
\inference[ESWhile\_ErrB ]{\judges{b, \sigma}{\bot}}
{\judges{\mathbf{while} \; b \; \mathbf{invariant} \; f \; \{ \; s \; \}, \sigma}{\bot}}
\end{equation*}
\begin{equation*}
\inference[ESWhile\_ErrS ]{\judges{b, \sigma}{true} & \judges{s, \sigma}{\bot}}
{\judges{\mathbf{while} \; b \; \mathbf{invariant} \; f \; \{ \; s \; \}, \sigma}{\bot}}
\end{equation*}
\caption{Semantics of statements in WHILE.}
\label{fig:stmtsemantics}
\end{figure}

Note that we have an $invariant$ in the syntax for while statements. This is only used when
proving correctness of a while loop, and not in the actual evaluation of a while statement.

