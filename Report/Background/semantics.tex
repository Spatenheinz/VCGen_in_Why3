We describe WHILE by its Natural semantics.
We use a store to keep the values of any program variables, and this is implemented using a finite map:
$$\sigma \in \Sigma = Var \longrightarrow \mathbb{Z} \cup \{\bot\}$$
where $Var$ defines all possible variables and $\bot$ denotes the result of unbound variables.
Hence we consider a store to be a total function and define a judgement for arithmetic expressions as
$\judge{a,\sigma}{(n | \bot)}$, where $n \in \mathbb{Z}$, $a$ is an arithmetic expression, and $x | y$ denotes either $x$ or $y$.
All the inference rules for arithmetic expressions are shown in \autoref{fig:aexprsemantics}.

\begin{figure}[h!]
\inference[ACst]{}
{\judge{n,\sigma}{n}}

\inference[Var]{}
{\judge{v,\sigma}{n}}[$(\sigma(v) = n)$]

\inference[Var\_Unbound]{}
{\judge{v,\sigma}{\bot}}[$( \sigma(v) = \bot )$]

\inference[BinOp]{\judge{a_{0}, \sigma}{n_{0}} & \judge{a_{1}, \sigma}{n_{1}}}
{\judge{a_{0} \oplus a_{1},\sigma}{n_{0} \oplus n_{1}}}

\inference[BinOp\_Unbound1]{\judge{a_{0}, \sigma}{\bot}}
{\judge{a_{0} \oplus a_{1},\sigma}{\bot}}

\inference[BinOp\_Unbound2]{\judge{a_{1}, \sigma}{\bot}}
{\judge{a_{0} \oplus a_{1},\sigma}{\bot}}

\label{fig:aexprsemantics}
\caption{Semantics of arithmetic expressions in WHILE.}
\end{figure}

In a similar manner we define a judgement for boolean expressions as $\judge{b,\sigma}{(t | \bot)}$, where $t \in \{true, false\}$ and $b$ is a boolean expression.
The semantics for boolean expressions are presented in \autoref{fig:bexprsemantics}.

\begin{figure}[h!]
\inference[BCst]{}
{\judge{t,\sigma}{t}}

\inference[NegT]{\judge{b,\sigma}{n_0} = true}
{\judge{ \neg b , \sigma}{false}}

\inference[NegF]{\judge{b,\sigma}{n_0} = false}
{\judge{ \neg b , \sigma}{true}}

\inference[Neg\_Unbound]{\judge{b,\sigma}{\bot}}
{\judge{ \neg b , \sigma}{\bot}}

\inference[LeqT]{\judge{a_0,\sigma}{n_0} & \judge{a_1,\sigma}{n_1}}
{\judge{a_0 \leq a_1 ,\sigma}{true}}[$(n_0 \leq n_1)$]

\inference[LeqF]{\judge{a_0,\sigma}{n_0} & \judge{a_1,\sigma}{n_1}}
{\judge{a_0 \leq a_1 ,\sigma}{true}}[$(n_0 > n_1)$]

\inference[Leq\_Unbound1]{\judge{a_0,\sigma}{\bot}}
{\judge{a_0 \leq a_1, \sigma}{\bot}}

\inference[Leq\_Unbound2]{\judge{a_1,\sigma}{\bot}}
{\judge{a_0 \leq a_1, \sigma}{\bot}}

\inference[AndT]{\judge{a_0, \sigma}{true} & \judge{a_1, \sigma}{true}}
{\judge{a_0 \land a_1, \sigma}{true}}

\inference[AndF0]{\judge{a_0, \sigma}{false}}
{\judge{a_0 \land a_1, \sigma}{false}}

\inference[AndF1]{\judge{a_0, \sigma}{true} & \judge{a_1, \sigma}{false}}
{\judge{a_0 \land a_1, \sigma}{false}}

\label{fig:bexprsemantics}
\caption{Semantics of boolean expressions in WHILE.}
\end{figure}

Likewise for statements a judgement will either be a new state, or an abnormal behaviour in terms of unbound variables: $\judge{s, \sigma}{(\sigma | \bot)}$, where $s$ is a statement.
The semantics of statements are presented in \autoref{fig:stmtsemantics}.

\begin{figure}[h!]

\inference[Skip]{}
{\judge{skip,\sigma}{\sigma}}

\inference[Assign]{\judge{a,\sigma}{n}}
{\judge{x := a,\sigma}{\sigma[x \mapsto n]}}

\inference[Assign\_Unbound]{\judge{a,\sigma}{\bot}}
{\judge{x := a,\sigma}{\bot}}

\inference[Assign\_Unbound]{\judge{a,\sigma}{\bot}}
{\judge{x := a,\sigma}{\bot}}

\inference[IfT]{\judge{b, \sigma}{true} & \judge{s_0, \sigma}{\sigma'}}
{\judge{\mathbf{if} \; b \; \{ \; s_0 \; \} \mathbf{else} \; \{ \; s_1 \; \}, \sigma}{\sigma'}}

\inference[IfF]{\judge{b, \sigma}{false} & \judge{s_1, \sigma}{\sigma'}}
{\judge{\mathbf{if} \; b \; \{ \; s_0 \; \} \mathbf{else} \; \{ \; s_1 \; \}, \sigma}{\sigma'}}

\inference[If\_UnboundB]{\judge{b, \sigma}{\bot}}
{\judge{\mathbf{if} \; b \; \{ \; s_0 \; \} \mathbf{else} \; \{ \; s_1 \; \}, \sigma}{\bot}}

\inference[If\_Unbound1]{\judge{b, \sigma}{true} & \judge{s_0, \sigma}{\bot}}
{\judge{\mathbf{if} \; b \; \{ \; s_0 \; \} \mathbf{else} \; \{ \; s_1 \; \}, \sigma}{\bot}}

\inference[If\_Unbound2]{\judge{b, \sigma}{false} & \judge{s_1, \sigma}{\bot}}
{\judge{\mathbf{if} \; b \; \{ \; s_0 \; \} \mathbf{else} \; \{ \; s_1 \; \}, \sigma}{\bot}}

\inference[WhileT]{\judge{b, \sigma}{true} & \judge{s_0, \sigma}{\sigma''} & \judge{\mathbf{while} b \; \{ \; s_0 \; \}, \sigma''}{\sigma'}}
{\judge{\mathbf{while} \; b \; \{ \; s_0 \; \}, \sigma}{\sigma'}}

\inference[WhileF]{\judge{b, \sigma}{false}}
{\judge{\mathbf{while} \; b \; \{ \; s_0 \; \}, \sigma}{\sigma}}

\inference[While\_UnboundB]{\judge{b, \sigma}{\bot}}
{\judge{\mathbf{while} \; b \; \{ \; s_0 \; \}, \sigma}{\bot}}

\inference[While\_UnboundS]{\judge{b, \sigma}{true} & \judge{s_0, \sigma}{\bot}}
{\judge{\mathbf{while} \; b \; \{ \; s_0 \; \}, \sigma}{\bot}}

\inference[Seq]{\judge{s_0, \sigma}{\sigma''} & \judge{s_1, \sigma''}{\sigma'}}
{\judge{s_0 \; ; \; s_1, \sigma}{\sigma'}}

\inference[Seq\_Unbound1]{\judge{s_0, \sigma}{\bot}}
{\judge{s_0 \; ; \; s_1, \sigma}{\bot}}

\inference[Seq\_Unbound2]{\judge{s_1, \sigma}{\bot}}
{\judge{s_0 \; ; \; s_1, \sigma}{\bot}}

\label{fig:stmtsemantics}
\caption{Semantics of statements in WHILE.}
\end{figure}

\textbf{TODO: update semantics to have invariants in while expressions.}

For assertions we simply cannot define the semantics in terms of operational semantics,
since the language contains quantifiers. We define it through a satisfaction relation.

\textbf{TODO: I am confused about this last part????}
see chapter3 of SATnotes.
