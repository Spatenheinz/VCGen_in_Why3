Why3 is a platform for deductive program verification initially released in 2012. Deductive program verification means that it can be used to prove properties of programs using a set of logical statements. The platform is vast and there are many ways to reason about programs.
The tool strives to be a standardized frontend with a high level of automation by using third-party provers, such as the aformentioned SMT-solvers and ATP, but also with the possibility to discharge to interactive theorem provers such as Coq and Isabelle.
For instance it is possible to use the tools as an intermediate tool to reason about C, Java programs etc. This can be done through the main language of the platform called WhyML.

In this project we use Why3 as a language for reasoning about the pbject language, meaning we
formalize the abstract syntax of our WHILE language directly in the WhuML.
As the name suggests, WhyML is a language in the ML family.
We will not go into too much detail about the specific syntax, for that we refer to the official documentation. We will however give a brief example of a program/function.

\begin{lstlisting}[caption={Function for evaluation of arithmetic expressions in Why3},label={lst:why3example},language=sml]
    let rec aeval (a: aexpr) : int
    variant { a }
    ensures { eval_aexpr a sigma (Eint result) }
    raises { E.Var_unbound -> eval_aexpr a sigma Eunbound }
  = match a with
    | Acst i -> i
    | Avar v ->  E.find sigma v
    | Asub a1 a2 -> aeval a1 - aeval a2
    | AMul a1 a2 -> aeval a1 - aeval a2
  end
\end{lstlisting}

The function \texttt{aeval} is a function which evaluates an arithmetic expression as defined in \autoref{lst:why3example}.
The body of the function is similar to whatever other ML dialect.
Where Why3 deviates a lot from other ML dialects is in the header of the function.
We define the function using \texttt{let rec}, which means that the function is recursive
and not necessarily pure.
There are many different function declarations but we will only ever be using function
declarations including a let, since any other declaration cannot be exported, and we want to
be able to extract the code into OCaml programs.
Furthermore the program contains a contract, consisting of 4 optional parts, namely the follwing:

\begin{enumerate}
  \item \texttt{variant}, which as previously described ensures total correctness of a program. The program may be any term with a well formed order. In the example function we use the arithmetic expression $a$.
  \item \texttt{requires}, serves as the precondition of the program. For this particular function we have no precondition.
  \item \texttt{ensures}, this is the postcondition of the program. In this case we state that the evaluation of an arithmetic expression should respect the semantics. The keyword \textit{result} refers to the result of evaluation.
  \item \texttt{raises}, states that when the function raises an exception the given predicate should hold. In particular we for the example function have that if a variable is unbound, and an exception is raised, then the evaluation results in a $\bot$ result, instead an integer.
\end{enumerate}

From this it should be clear that Why3 tries to make the gap for program verification as small
as possible, by allowing program constructs and controlflows familiar to ``regular''
programming languages, while also having ways of including assertions etc in the code.

One can then use either the Why3 IDE or the terminal interface to discharge the verification condition for the function. To see how a proof can be discharged see Appendix \ref{run}.

Not only does whyML allow one to reason about a program through its verification condition,
but it is also possible to reason about programs through the embedded logic-language
(and strictly speaking the contract from before is part of that logic language).
For instance it is possible to define predicates and, even more powerful, lemmas about programs.
A lemma might look as follows:

\begin{lstlisting}[caption={Lemma stating determinism of arithmetic expressions},label={lst:why3fun},language=sml]
  lemma eval_aexpr_fun : forall a s b1 b2.
     eval_aexpr a s b1 -> eval_aexpr a s b2 -> b1 = b2
   \end{lstlisting}

This lemma states that forall arithmetics expressions \texttt{a}, stores \texttt{s}, results \texttt{b1} and \texttt{b2} the semantics of arithmetic expressions ensures that \texttt{b1} and \texttt{b2} are the same.
In other words the function is deterministic. After trying to prove this lemma, whether is proves or not, it will be ``saved'' as an axiom and can be used in following lemmas.
It should furthermore be noted that WhyML is restricted to first order logic to only allow proof-obligations to be of first order logic.
\\~\\
So, by using Why3 we can reason about WHILE and wlp on a high level and by
correctness-by-construction we can extract a program to either C or OCaml code.
We will in the later sections see how well the claim of high level of automation holds up, and discuss the possibility of extracting code from Why3 programs.
