% Some stuff about wp
Now that we have looked at a general axiomatic semantic of our WHILE language, we will look at \textfit{weakest precondition calculus}, which is another sort of axiomatic semantic for our language.
Weakest precondition calculus is used for generating verification condition, as we can use this calculus to determine the weakest precondition of a program, ie. the weakest precondtition that must hold for the post condition to hold.
This precondition can then be used as a verification condition for the program, meaning that if the weakest precondition holds, then the correctness of the program can be assured. However, the weakest precondition also assures termination, and that is not always possible. When we are interested in asserting correctness, but not termination, we use \textit{weakest liberal precondition}.


Now we present the rules for determining the weakest liberal precondition of an expression.
Next we will present the soundness etc of this system ??? \textbf{TODO}.

\paragraph{Rules for determining weakest liberal precondition}
% rules of wp for the construct of our WHILE language
The weakest liberal precondition is the minimal condition that must hold to prove correctness of a program, assuming that it terminates.
The \textit{wlp} for our WHILE language (see \autoref{sec:syntax} for syntax) is determined using the rules presented in \autoref{fig:wlp}.

\begin{figure}
\begin{align*}
WLP(\texttt{skip}, Q) &= Q \\
WLP(x:=e,Q) &= \forall y, y = e \Rightarrow Q[x \leftarrow y] \\
WLP(s_0;s_1, Q) &= WLP(s_0, WLP(s_1, Q)) \\
WLP(\texttt{if } e \texttt{ then } s_0 \texttt{ else } s_1, Q) &= (e \Rightarrow WLP(s_0, Q)) \\
    &\quad \land (\neg e \Rightarrow WLP(s_1, Q)) \\
WLP(\texttt{while } e \texttt{ invariant } I \texttt{ do } s, Q) &=
    I \land \\
&\forall x_1, ..., x_k, \\
&(((e \land I) \Rightarrow WLP(s, I)) \\
&\quad \land (( \neg e \land I) \Rightarrow Q))
    [w_i \leftarrow x_i] \\
&\text{where } w_1, ..., w_k \text{ is the set of assigned variables in} \\
&\text{statement } s \text{ and } x_1, ..., x_k \text{ are fresh logical variables.} \\
WLP(\{P\}, Q) &= P \land Q \quad \text{where P is an assertion}
\end{align*}
\caption{Rules for computing weakest liberal precondition \cite{wlp}.}
\label{fig:wlp}
\end{figure}

The rules define how to compute the $wlp$ when given a statement $s$ and a formula $Q$, meaning that it is a function $WLP(s,Q)$ which outputs the $wlp$.
We will now briefly describe each rule.

The $wlp$ of \texttt{skip} is just the formula $Q$, as the statement has no effect.
The $wpl$ of \texttt{x:=e} with $Q$ is simply $Q$ where each occurence of $x$ is exchanged with $e$, as that is the effect of the assignment.
The $wlp$ of \texttt{s_0;s_1} is the result of first determining $WLP(s_{1},Q) = Q_{1}$, and then using that result to compute the overall $wlp$ as $WLP(s_{0}, Q_{1})$. This is because we always compute the $wlp$ by looking at the last statement, building up a formula from right to left.

The next two rules are more interesting.
For determining the $wlp$ of an \texttt{if} statement, the rule states that \textit{if} $e$ is true, then the $wlp$ is $WLP(s_{0}, Q)$, else it is $WLP(s_{1},Q)$, thus choosing a branch according to the condition.
Next we have the rule for determining the $wlp$ of \texttt{while} statements.
It states that firstly the invariant $I$ must hold. Secondly we want to exchange all assigned variables $w_{0},...,w_{k}$ with the fresh logical variables $x_{0},...,x_{k}$. Next we want assert that if both the invariant $I$ and the condition $e$ holds, then $WLP(s, I)$ must hold, as we execute the loop body, and the invariant must be true in the end of each loop iteration. If $e$ doesn not hold, but $I$ does, then we do not loop again, and the statement has the same effect as a \texttt{skip} statement, thus we assert that $Q$ holds in this case.

The last rule concerns assertion statements, and this case is trivial, as we simply add the assertion to the formula $Q$ by conjuntion.

\paragraph{Soundness etc}
% soundness etc. refer to automated proofs in section ???
