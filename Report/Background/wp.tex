% Some stuff about wp
Now that we have looked at a general axiomatic semantic of our WHILE language, we will look at \textfit{weakest precondition calculus}, which is another sort of axiomatic semantic for our language.

$WP$ was first introduced by Edgar W. Dijkstra and is defined as follows

$$
\forall P. P \Longrightarrow Q \; \text{iff} \; \hoare{P}{s}{Q} \; \text{is valid}
$$

That is $P$ is a weakest precodition for $s$ such that $Q$ will hold in the halting state of $s$.

We use weakest precondition calculus for generating verification conditions, as we can use
this calculus to determine the weakest precondition of a program, ie. the weakest precondition
that must hold before a statement for the postcondition to hold afte the execution of the statement.
This precondition can then be used as a verification condition for the program, meaning that if the weakest precondition holds, then the correctness of the program can be assured.
However, the weakest precondition also assures termination, and that is not always possible. When we are interested in asserting correctness, but not termination, we use \textit{weakest liberal precondition}.
We denote the weakest liberal precodition as $P = wlp(s,Q)$.

Now we present the rules for determining the weakest liberal precondition of an expression.
Next we will present the intuition behind proving the soundness of this system.

\paragraph{Rules for determining weakest liberal precondition}
% rules of wp for the construct of our WHILE language
The weakest liberal precondition is the minimal condition that must hold to prove correctness of a program, assuming that it terminates.
The \textit{wlp} for our WHILE language (see \ref{sec:syntax} for syntax) is determined using the rules presented in \autoref{fig:wlp}.

\begin{figure}[h!]
\begin{align*}
WLP(\texttt{skip}, Q) &= Q \\
WLP(x:=e,Q) &= \forall y, y = e \Rightarrow Q[x \mapsto y] \\
WLP(s_0;s_1, Q) &= WLP(s_0, WLP(s_1, Q)) \\
WLP(\{P\}, Q) &= P \land Q \quad \text{where P is an assertion} \\
WLP(\texttt{if } e \texttt{ then } s_0 \texttt{ else } s_1, Q) &= (e \Rightarrow WLP(s_0, Q)) \\
    &\quad \land (\neg e \Rightarrow WLP(s_1, Q)) \\
WLP(\texttt{while } e \texttt{ invariant } I \texttt{ do } s, Q) &=
    I \land \\
&\forall x_1, ..., x_k, \\
&(((e \land I) \Rightarrow WLP(s, I)) \\
&\quad \land (( \neg e \land I) \Rightarrow Q))
    [w_i \mapsto x_i] \\
&\text{where } w_1, ..., w_k \text{ is the set of assigned variables in} \\
&\text{statement } s \text{ and } x_1, ..., x_k \text{ are fresh logical variables.}
\end{align*}
\caption{Rules for computing weakest liberal precondition \cite{wlp}.}
\label{fig:wlp}
\end{figure}

The rules define how to compute the $wlp$ when given a statement $s$ and a formula $Q$, meaning that it is a function $WLP(s,Q)$ which outputs the $wlp$.
We will now briefly describe each rule.

The $wlp$ of \texttt{skip} is just the formula $Q$, as the statement has no effect.
The $wpl$ of \texttt{x:=e} with $Q$ is simply $Q$ where each occurence of $x$ is exchanged with $e$, as that is the effect of the assignment.
The $wlp$ of \texttt{s_0;s_1} is the result of first determining $WLP(s_{1},Q) = Q_{1}$, and then using that result to compute the overall $wlp$ as $WLP(s_{0}, Q_{1})$. This is because we always compute the $wlp$ by looking at the last statement, building up a formula from right to left.

The next two rules are more interesting.
For determining the $wlp$ of an \texttt{if} statement, the rule states that \textit{if} $e$ is true, then the $wlp$ is $WLP(s_{0}, Q)$, else it is $WLP(s_{1},Q)$, thus choosing a branch according to the condition.
Next we have the rule for determining the $wlp$ of \texttt{while} statements.
It states that firstly the invariant $I$ must hold. Secondly we want to exchange all assigned variables $w_{0},...,w_{k}$ with the fresh logical variables $x_{0},...,x_{k}$. Next we want assert that if both the invariant $I$ and the condition $e$ holds, then $WLP(s, I)$ must hold, as we execute the loop body, and the invariant must be true in the end of each loop iteration. If $e$ doesn not hold, but $I$ does, then we do not loop again, and the statement has the same effect as a \texttt{skip} statement, thus we assert that $Q$ holds in this case.

The last rule concerns assertion statements, and this case is trivial, as we simply add the assertion to the formula $Q$ by conjunction.

\paragraph{Soundness of wlp}
% soundness etc. refer to automated proofs in section ???
For the weakest liberal precondition calculus to be meaningful to us, we must assert that the calculus is actually sound.
This is done by proving that \textit{for all statements s and formula Q, \{WLP(s, Q)\}s\{Q\} is valid for partial correctness}.
We present here the stategy and intuition behind the proof, using the prcedure presented in \cite{wlp}.

The proof is done by considering the rules of finding the wlp for the different statements, thus it uses structural induction on $s$.
As a preliminary remark it should be noted that for any formula $\phi$ and any state $\Sigma$, we have that the interpretation of $\forall x_{1} ... x_{k} \phi [w_{i} \mapsto x_{i}]$ in $\Sigma$ does not depend on the values of variables in $\Sigma$, thus if it holds for one state $\Sigma$ then it should hold for all states $\Sigma'$ that only differs from $\Sigma$ in the value of the variables.
\\~\\
Now lets look at the structure of the proof.
We consider each possible case of $s$, thus considering all the different types of statements.
Most of these are straightforward, but the case for $while$-expressions is somewhat tricky.
Therefore we will dive a bit into the intuition behind that case.

Lets assume that $s =$ \texttt{while $cond$ invariant $inv$ $body$}.
Now we want to show that \{WLP(s, Q)\}s\{Q\} holds. We assume that we have a state $\Sigma$ such that $\llbracket WLP(s, Q)\rrbracket_{\Sigma}$ holds, and that $s$ executes in $\Sigma$ and terminates, given by $\Sigma, s \rightsquigarrow^{*} \Sigma '$.
The rest of the proof uses induction on the length of the execution denoted by $\rightsquigarrow^{*}$.

\subparagraph{Case $e=0$.}
This terminates the loop, thus $\Sigma = \Sigma'$.
In this case we have that $\llbracket e \rrbracket_{\Sigma} = 0$, by the assumption.
From the wlp rule for while expressions, we get that $\llbracket I \rrbracket_{\Sigma}$ and $\llbracket ((e = 0 \land I) \Rightarrow Q) [ w_{i} \mapsto x_{i}] \rrbracket_{\Sigma}$.
By the preliminary notion, we know that we can instatiate each $w_{i}$ to the variables of $\Sigma$, and then we get $\llbracket (e = 0 \land I) \Rightarrow Q \rrbracket_{\Sigma}$.
Now, as $\llbracket I \rrbracket_{\Sigma}$ and $\llbracket e \rrbracket_{\Sigma} = 0$ we get that $\llbracket Q \rrbracket_{\Sigma}$, which was exactly what we wanted to show.

\subparagraph{Case $e \neq 0 $.}
In this case the loop does not terminate, but executes another iteration of the body.
That means $\Sigma \neq \Sigma'$, and that the execution must first execute $\Sigma, s \rightsquigarrow \Sigma$, as the state does not change at first.
Then the execution must execute the sequence $b;s$ in $\Sigma$, and end in some state $\Sigma''$.
Lastly, from that state executing $s$ again will get us to the final state $\Sigma$.

\textbf{TODO: explain this better}.

We can prove that $\{WLP(s, Q)\}s\{Q\}$ by proving that $\llbracket I \rrbracket_{\Sigma''}$ and $\llbracket WLP(s, Q)\rrbracket_{\Sigma''}$, from where we can use induction on the length of the execution to say that $\llbracket Q \rrbracket_{\Sigma'}$.
By the assumption we know that $\llbracket e \rrbracket_{\Sigma} \neq 0$, and by the wlp rule we have that $\llbracket I \rrbracket_{\Sigma}$ and $\llbracket ((e \neq 0 \land I) \Rightarrow WLP(b, I)) [ w_{i} \mapsto x_{i}] \rrbracket_{\Sigma}$.
Again we instatiate all variables to those in $\Sigma$, and get $\llbracket ((e \neq 0 \land I) \Rightarrow WLP(b, I)) \rrbracket_{\Sigma}$.
Because we have that $\llbracket e \rrbracket_{\Sigma} \neq 0$ and $\llbracket I \rrbracket_{\Sigma}$, we get that $\llbracket WLP(b, I) \rrbracket_{\Sigma}$.
First we need to know that $I$ holds in $\Sigma''$. By IH we have $\{WLP(b, I)\}b\{I\}$, which means that $\llbracket I \rrbracket_{\Sigma''}$.

As described in \cite{wlp} the only difference between $\Sigma''$ and $\Sigma$ is the values of the variables $w_{i}$, and by the preliminary remark we get that $\llbracket ((e \neq 0 \land I) \Rightarrow WLP(b, I)) [ w_{i} \mapsto x_{i}] \rrbracket_{\Sigma''}$, which means that we now have all we need to know that $\llbracket WLP(s, Q) \rrbracket_{\Sigma''}$.
We now use the induction on the length of the execution to say that $\llbracket Q \rrbracket_{\Sigma'}$, which is exactly what we wanted to show.
