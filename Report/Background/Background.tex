\section{Background}
When writing a verified VC generator, we must consider both the language that we want to interpret, the syntax and semantics, both operational and axiomatic, of the object language, and the language that we use for writing the VC generator in.

Firstly, we consider an object language \textbf{WHILE} which will be the input for the compiler.
In \autoref{sec:while} we describe the syntax and semantics of the object language.

Secondly, we consider the axiomatic semantics of the language, which we use for generating verification conditions for a given program. This is presented in \autoref{sec:hoarewp}, where we describe the Hoare logic and weakest precondiion calculus that we build VC generation on.

Thirdly, we consider the host language used for writing the compiler.
We use Why3 and Isabelle as our two host languages. A brief introduction to both is given in \autoref{sec:why3} and \autoref{sec:isabelle} respectively.

\subsection{WHILE}\label{sec:while}
In this project we work with a simple WHILE language, which is designed to be a minimal but still expressive language.
In \autoref{sec:syntax} we define the syntax of the language, and in \autoref{sec:semantics} the operational semantics.

\subsubsection{Syntax}\label{sec:syntax}
As object language we consider a simple imperative language often in the literature referred to as either
IMP or WHILE and from hereon out we will reference our definition as the latter. Notice the language might differ slightly from other definitions but any such derivations should be insignificant.
We consider a language with two basic types, integers and booleans.
\autoref{fig:grammaraexpr} shows the grammar for arithmetic and boolean expressions.
An arithmetic expression might either be a variable, integer, a substitution of two variables.
Notice here than we only define a single binary operation for arithmetic expressions.
We do so for a simpler compiler, without loss of expressiveness as we as syntactic sugar can define addition as
$a_{0} + a_{1} = a_{0} - (0 - a_{1})$.

In a similar manner we can define boolean expressions as either $\top$ (true), $\bot$ (false), negation, conjunction
and relational operators. again we limit relational operators to only a single operator as we can describe the rest in terms of $\leq$ paired with $\neg$.

\colorbox{BurntOrange}{TODO???? We might make a table of the syntactic sugared versions.}

Likewise with the functionally complete set $\{\wedge, \neg\}$ we can define all possible boolean expressions.

\begin{figure}[h!]
\centering
\begin{grammar}
<aexpr> ::= <identifier>
\alt <integer>
\alt <aexpr> <aop> <aexpr>
\alt '(' <aexpr> ')'

<aop> ::= '$-$'

<bexpr> ::= 'true' | 'false' | \alt '$\neg$'<bexpr>
\alt <bexpr> '$\wedge$' <bexpr>
\alt <aexpr> <rop> <aexpr>
\alt '(' <bexpr> ')'

<rop> ::= '$\le$'
\end{grammar}
\label{fig:grammaraexpr}
\caption{Grammar for arithmetic \& boolean expressions}
\end{figure}

In a similar manner we define assertions corresponding to a first order logic. Assertions
will be used for two things. Mainly this will be used for the Hoare logic and Verification condition generation,
but secondarily as regular assertions, which will terminate program execution if the assertion is not valid.
We have a functionally complete set in $\{\forall, \wedge, \neg\}$.

Note here that in other definitions of assertion languages in relation to hoare logic it is common to include ghost variables, which are not allowed in the execution of programs and hence are only allowed in the logic language. In our current state, we do not include such variables.
\begin{figure}[h!]
\centering
\begin{grammar}
<assertion> ::= '$\forall$' <identifier> '.' <assertion>
\alt '$\neg$' <assertion>
\alt <assertion> '$\wedge$' <assertion>
\alt <bexpr>
\end{grammar}
\label{fig:grammaraexpr}
\caption{Grammar of assertion language}
\end{figure}

A WHILE program is a possibly empty sequence of statements. a statement can either be
a variable binding, if-else statement, a while loop, an assertion or a skip.

\begin{figure}[h!]
\centering
\begin{grammar}
<statements> ::= <statement> ';' <statements> | $\epsilon$

<statement> ::= <identifier> ':=' <aexpr>
\alt 'if' <bexpr> '\{' <statements> '\}' 'else' '\{' <statements> '\}'
\alt 'while' <bexpr> '\{' <statements> '\}'
\alt '\#\{' <assertion> '\}'
\alt 'skip'
\end{grammar}
\label{fig:grammaraexpr}
\caption{Grammar for a WHILE program}
\end{figure}

In the future it would be good to consider additional types but we save this for future work.
\subsubsection{Semantics}\label{sec:semantics}
We describe WHILE by its Natural semantics.
we use a store to keep the values of any program variables. We use a finite map:
$$\sigma \in \Sigma = Var \longrightarrow \mathbb{Z} \cup \{\bot\}$$
where $Var$ defines all possible variables and $\bot$ denotes the result of unbound variables.
Hence we consider a store to be a total function and define a judgement for arithmetic expressions as
$\judge{a,\sigma}{(n | \bot)}$, where $n \in \mathbb{Z}$ $a$ is an arithmetic expression and $x | y$ denotes either $x$ or $y$.

\begin{figure}[h!]
\inference[ACst]{}
{\judge{n,\sigma}{n}}

\inference[Var]{}
{\judge{v,\sigma}{n}}[$(\sigma(v) = n)$]

\inference[Var\_Unbound]{}
{\judge{v,\sigma}{\bot}}[$( \sigma(v) = \bot )$]

\inference[BinOp]{\judge{a_{0}, \sigma}{n_{0}} & \judge{a_{1}, \sigma}{n_{1}}}
{\judge{a_{0} \oplus a_{1},\sigma}{n_{0} \oplus n_{1}}}

\inference[BinOp\_Unbound1]{\judge{a_{0}, \sigma}{\bot}}
{\judge{a_{0} \oplus a_{1},\sigma}{\bot}}

\inference[BinOp\_Unbound2]{\judge{a_{1}, \sigma}{\bot}}
{\judge{a_{0} \oplus a_{1},\sigma}{\bot}}
\end{figure}

In a similar manner we define a judgement for boolean expressions as $\judge{b,\sigma}{(t | \bot)}$, where $t \in \{true, false\}$ and $b$ is a boolean expression.

\begin{figure}[h!]
\inference[BCst]{}
{\judge{t,\sigma}{t}}

\inference[LeqT]{\judge{a_0,\sigma}{n_0} & \judge{a_1,\sigma}{n_1}}
{\judge{a_0 \leq a_1 ,\sigma}{true}}[$(n_0 \leq n_1)$]

\inference[LeqF]{\judge{a_0,\sigma}{n_0} & \judge{a_1,\sigma}{n_1}}
{\judge{a_0 \leq a_1 ,\sigma}{true}}[$(n_0 > n_1)$]

\inference[Leq\_Unbound1]{\judge{a_0,\sigma}{\bot}}
{\judge{a_0 \leq a_1, \sigma}{\bot}}

\inference[Leq\_Unbound2]{\judge{a_1,\sigma}{\bot}}
{\judge{a_0 \leq a_1, \sigma}{\bot}}

\inference[NegT]{\judge{b,\sigma}{n_0} = true}
{\judge{ \neg b , \sigma}{false}}

\inference[NegF]{\judge{b,\sigma}{n_0} = false}
{\judge{ \neg b , \sigma}{true}}

\inference[Neg\_Unbound]{\judge{b,\sigma}{\bot}}
{\judge{ \neg b , \sigma}{\bot}}

\end{figure}

Likewise for statements a judgement will either be an abnormal behaviour in terms of
unbound variables: $\judge{s, \sigma}{(\sigma | \bot)}$, where $s$ is a statement.

\begin{figure}[h!]

\inference[Skip]{}
{\judge{skip,\sigma}{\sigma}}

\inference[Assign]{\judge{a,\sigma}{n}}
{\judge{x := a,\sigma}{\sigma[x \mapsto n]}}

\inference[Assign\_Unbound]{\judge{a,\sigma}{\bot}}
{\judge{x := a,\sigma}{\bot}}

\inference[Assign\_Unbound]{\judge{a,\sigma}{\bot}}
{\judge{x := a,\sigma}{\bot}}

\inference[IfT]{\judge{b, \sigma}{true} & \judge{s_0, \sigma}{\sigma'}}
{\judge{\mathbf{if} \; b \; \{ \; s_0 \; \} \mathbf{else} \; \{ \; s_1 \; \}, \sigma}{\sigma'}}

\inference[IfF]{\judge{b, \sigma}{false} & \judge{s_1, \sigma}{\sigma'}}
{\judge{\mathbf{if} \; b \; \{ \; s_0 \; \} \mathbf{else} \; \{ \; s_1 \; \}, \sigma}{\sigma'}}

\inference[If\_UnboundB]{\judge{b, \sigma}{\bot}}
{\judge{\mathbf{if} \; b \; \{ \; s_0 \; \} \mathbf{else} \; \{ \; s_1 \; \}, \sigma}{\bot}}

\inference[If\_Unbound1]{\judge{b, \sigma}{true} & \judge{s_0, \sigma}{\bot}}
{\judge{\mathbf{if} \; b \; \{ \; s_0 \; \} \mathbf{else} \; \{ \; s_1 \; \}, \sigma}{\bot}}

\inference[If\_Unbound2]{\judge{b, \sigma}{false} & \judge{s_1, \sigma}{\bot}}
{\judge{\mathbf{if} \; b \; \{ \; s_0 \; \} \mathbf{else} \; \{ \; s_1 \; \}, \sigma}{\bot}}

\inference[WhileT]{\judge{b, \sigma}{true} & \judge{s_0, \sigma}{\sigma''} & \judge{\mathbf{while} b \; \{ \; s_0 \; \}, \sigma''}{\sigma'}}
{\judge{\mathbf{while} \; b \; \{ \; s_0 \; \}, \sigma}{\sigma'}}

\inference[WhileF]{\judge{b, \sigma}{false}}
{\judge{\mathbf{while} \; b \; \{ \; s_0 \; \}, \sigma}{\sigma}}

\inference[While\_UnboundB]{\judge{b, \sigma}{\bot}}
{\judge{\mathbf{while} \; b \; \{ \; s_0 \; \}, \sigma}{\bot}}

\inference[While\_UnboundS]{\judge{b, \sigma}{true} & \judge{s_0, \sigma}{\bot}}
{\judge{\mathbf{while} \; b \; \{ \; s_0 \; \}, \sigma}{\bot}}

\inference[Seq]{\judge{s_0, \sigma}{\sigma''} & \judge{s_1, \sigma''}{\sigma'}}
{\judge{s_0 \; ; \; s_1, \sigma}{\sigma'}}

\inference[Seq\_Unbound1]{\judge{s_0, \sigma}{\bot}}
{\judge{s_0 \; ; \; s_1, \sigma}{\bot}}

\inference[Seq\_Unbound2]{\judge{s_1, \sigma}{\bot}}
{\judge{s_0 \; ; \; s_1, \sigma}{\bot}}

\end{figure}

For assertions we simply cannot define the semantics in terms of operational semantics,
since the language contains quantifiers. We define it through a satisfaction relation.

see chapter3 of SATnotes.

\subsection{Hoare logic and WP calculus}\label{sec:hoarewp}

\subsubsection{Hoare logic}\label{sec:hoare}
Floyd–Hoare logic (from now simply refered to as Hoare logic) is a formal system proposed by Tony Hoare in 1969,
based on earlier work from Robert W, which is used to prove correctness of a program.
The core of this formal system is a Hoare triple denoted as $\hoarePQ{s}$
where $P$ and $Q$ are assertions. This notion means that if the precondition $P$ holds in the initial state
and $s$ terminates then $Q$ holds in the halting state of $s$. More formally we write:

$$\hoarePQ{s} = \forall \sigma, \sigma' \in \Sigma. (\sigma \vDash P) \wedge ( \judge{s, \sigma}{\sigma'} \rightarrow (\sigma' \vDash Q))$$

Thus if a hoare triple holds, it is correct wrt to the assertions.

Hoare logic provides axioms and inference rules for an imperative language.
The statements included in these rules corresponds to statements provided in \autoref{sec:syntax}-\autoref{sec:semantics}. \autoref{fig:hoare} shows the inference rules.
Notice that we show the formalization in the traditional sense and do not consider behaviour of unbound variables.

\begin{figure}[h!]

\inference[HSkip]{}
{\hoarePQ{\mathbf{skip}}}

\inference[HAssign]{}
{\hoare{Q[x \mapsto a]}{x := a}{Q}}

\inference[HIf]{\hoare{P \wedge b}{s_0}{Q} & \hoare{P \wedge \neg b}{s_1}{Q}}
{\hoarePQ{ \mathbf{if} \; b \; \{ \; s_0 \; \} \mathbf{else} \; \{ \; s_1 \; \} }}

\inference[HWhile]{\hoare{P \wedge b}{s}{P}}
{\hoare{P}{ \mathbf{while} \; b \; \{ \; s \; \} }{P \wedge \neg b}}

\inference[HSeq]{\hoare{P}{s_0}{R} & \hoare{R}{s_1}{Q}}
{\hoarePQ{s_0 \; ; \; s_1}}

\inference[HCons]{\vDash P \rightarrow P' \hoare{P}{s_0}{Q} & \vDash Q' \rightarrow Q}
{\hoarePQ{s_0}}

\label{fig:hoare}
\caption{Hoare logic inference rules.}
\end{figure}

The rules are rather intuitive.
For Hskip, the pre and postcondition must be the same.
For Hassign all occurences of x in postcondition Q must be substituted with a.
For conditionals HIf we have that if the precondition P and the conditional check b holds then Q must hold after the then branch $s_{0}$, and if P and not b holds then Q must hold after the else branch $s_1$.
For while the assertion, P is called the invariant and must hold before each iteration of the loop and after the loop. In a similar fashion if the conditional b holds before an iteration of s then the invariant must still hold. And when the loop ends, the condition must not hold any longer.
Sequencing is rather trivial, if P holds before s_0 then R must hold after s_1 Q must hold.
Lastly the conseqeuce rule HCons is used to strengthen a precondition or weaken a postcondtion.
From these rules many additional statements can be derived.
\\~\\
Using these inference rules one can show a program to be partially correct.
The reason we can only show partial correctness is because the transitional sequence of statements might be infinite, i.e. a while loop might not terminate.

MAYBE WRITE ABOUT FINITE TRANSITIVE CLOSURE

To prove total correctness additional information for while-loops have to be included in the logic.
To prove termination a variant $v$ must be provided and it must be subject to a well-founded relation $\prec$.
\[
\inference[HWhile]{\hoare{P \wedge b \wedge v = \xi }{s}{P \wedge \xi}}
{\hoare{P}{ \mathbf{while} \; b \; \mathbf{invariant} \; P \; \mathbf{variant} v, \prec \; \{ \; s \; \} }{P \wedge \neg b}}[wf(\prec)]
\]
where, $\xi$ is a fresh variable. By this total correctness can be achieved.
\\~\\
Hoare logic can be shown to be sound, that is if a Hoare triple is proveable from the inference rules then the hoare triple is valid, i.e. $\vdash \hoarePQ{s}$ implies $\vDash \hoarePQ{s}$. This is an important
property, since we will be unable to derive partial correctness proofs that does not hold. The proof is done on the induction on the derivation of $\vdash \hoarePQ{s}$. We omit this proof, because we dont use Hoare logic in our formalization directly but rather as a consequence for our Verification condition generation.
\\~\\
On the other hand Hoare logic was in 1974\cite{} shown to be relatively complete with respect to the assertion language used. That is hoare logic is no more incomplete than the assertion language. This means that $\vdash \hoarePQ{s}$ implies $\vDash \hoarePQ{s}$ and with an expressive enough language.
Notice furthermore that without multiplication Hoare logic would not be complete.

The proof Cook provided used the notion of Weakest liberal precondition (wlp). Wlp was first introduced by Edgar W. Dijkstra and is defined as follows

$$
\forall P'. P' \Longrightarrow P \; \text{iff} \; \hoare{P'}{s}{Q} \; \text{is valid}
$$

That is $P$ is a weakest precodition for P such that $Q$ will hold in the halting state of $s$. We denote
the weakest liberal precodition as $P = wlp(s,Q)$.

\subsubsection{Weakest Precondition Calculus}\label{sec:wp}
% Some stuff about wp
Now that we have looked at a general axiomatic semantic of our WHILE language, we will look at \textfit{weakest precondition calculus}, which is another sort of axiomatic semantic for our language.

$WP$ was first introduced by Edgar W. Dijkstra and is defined as follows

$$
\forall P. P \Longrightarrow Q \; \text{iff} \; \hoare{P}{s}{Q} \; \text{is valid}
$$

That is $P$ is a weakest precodition for $s$ such that $Q$ will hold in the halting state of $s$.

We use weakest precondition calculus for generating verification conditions, as we can use
this calculus to determine the weakest precondition of a program, ie. the weakest precondition
that must hold before a statement for the postcondition to hold afte the execution of the statement.
This precondition can then be used as a verification condition for the program, meaning that if the weakest precondition holds, then the correctness of the program can be assured.
However, the weakest precondition also assures termination, and that is not always possible. When we are interested in asserting correctness, but not termination, we use \textit{weakest liberal precondition}.
We denote the weakest liberal precodition as $P = wlp(s,Q)$.

Now we present the rules for determining the weakest liberal precondition of an expression.
Next we will present the intuition behind proving the soundness of this system.

\paragraph{Rules for determining weakest liberal precondition.}
% rules of wp for the construct of our WHILE language
The weakest liberal precondition is the minimal condition that must hold to prove correctness of a program, assuming that it terminates.
The \textit{wlp} for our WHILE language (see \ref{sec:syntax} for syntax) is determined using the rules presented in \autoref{fig:wlp}.

\begin{figure}[h!]
\begin{align*}
WLP(\texttt{skip}, Q) &= Q \\
WLP(x:=a,Q) &= \forall y, y = a \Rightarrow Q[x \mapsto y] \\
WLP(s_0;s_1, Q) &= WLP(s_0, WLP(s_1, Q)) \\
WLP(\{P\}, Q) &= P \land Q \quad \text{where P is an assertion} \\
WLP(\texttt{if } b \texttt{ then } s_0 \texttt{ else } s_1, Q) &= (b \Rightarrow WLP(s_0, Q)) \\
    &\quad \land (\neg b \Rightarrow WLP(s_1, Q)) \\
WLP(\texttt{while } b \texttt{ invariant } I \texttt{ do } s, Q) &=
    I \land \\
&\forall x_1, ..., x_k, \\
&(((b \land I) \Rightarrow WLP(s, I)) \\
&\quad \land (( \neg b \land I) \Rightarrow Q))
    [w_i \mapsto x_i] \\
&\text{where } w_1, ..., w_k \text{ is the set of assigned variables in} \\
&\text{statement } s \text{ and } x_1, ..., x_k \text{ are fresh logical variables.}
\end{align*}
\caption{Rules for computing weakest liberal precondition \cite{wlp}.}
\label{fig:wlp}
\end{figure}

The rules define how to compute the $wlp$ when given a statement $s$ and a formula $Q$, meaning that it is a function $WLP(s,Q)$ which outputs the $wlp$.

The $wlp$ of \texttt{skip} is just the formula $Q$, as the statement has no effect.
The $wpl$ of \texttt{x:=e} with $Q$ is simply $Q$ where each occurence of $x$ is exchanged with $e$, as that is the effect of the assignment.
The $wlp$ of $s_0;s_1$ is the result of first determining $WLP(s_{1},Q) = Q_{1}$, and then using that result to compute the overall $wlp$ as $WLP(s_{0}, Q_{1})$. This is because we always compute the $wlp$ by looking at the last statement, building up a formula from right to left.

The next two rules are more interesting.
For determining the $wlp$ of an \texttt{if} statement, the rule states that \textit{if} $e$ is true, then the $wlp$ is $WLP(s_{0}, Q)$, else it is $WLP(s_{1},Q)$, thus choosing a branch according to the condition.
Next we have the rule for determining the $wlp$ of \texttt{while} statements.
It states that firstly the invariant $I$ must hold. Secondly we want to exchange all assigned variables $w_{0},...,w_{k}$ with the fresh logical variables $x_{0},...,x_{k}$. Next we want assert that if both the invariant $I$ and the condition $e$ holds, then $WLP(s, I)$ must hold, as we execute the loop body, and the invariant must be true in the end of each loop iteration. If $e$ doesn not hold, but $I$ does, then we do not loop again, and the statement has the same effect as a \texttt{skip} statement, thus we assert that $Q$ holds in this case.

The last rule concerns assertion statements, and this case is trivial, as we simply add the assertion to the formula $Q$ by conjunction.

\paragraph{Soundness of wlp.}
% soundness etc. refer to automated proofs in section ???
For the weakest liberal precondition calculus to be meaningful to us, we must assert that the calculus is actually sound.
This is done by proving that \textit{for all statements s and formula $Q$, $\{WLP(s, Q)\}s\{Q\}$ is valid for partial correctness}.

We present here the stategy and intuition behind the proof, using the prcedure presented in \cite{wlp}.

The proof is done by considering the rules of finding the wlp for the different statements, thus it uses structural induction on $s$.
As a preliminary remark it should be noted that for any formula $\phi$ and any state $\sigma$, we have that the interpretation of $\forall x_{1} ... x_{k} . \phi [w_{i} \mapsto x_{i}]$ in $\sigma$ does not depend on the values of variables in $\sigma$, thus if it holds for one state $\sigma$ then it should hold for all states $\sigma'$ that only differs from $\sigma$ in the value of the variables.
\\~\\
Now lets look at the structure of the proof.
We consider each possible case of $s$, thus considering all the different types of statements.
Most of these are straightforward, but the case for $while$-expressions is somewhat tricky.
Therefore we will dive a bit into the intuition behind that case.

Lets assume that $s =$ \texttt{while $b$ invariant $I$ $s'$}.
Now we want to show that $\{WLP(s, Q)\}s\{Q\}$ holds for any $Q$.
Assume that we have a state $\sigma$ such that $\sigma \vDash WLP(s, Q)$ holds, and that $\judges{s,\sigma}{\sigma'}$

The rest of the proof uses induction on the length of the execution.

\subparagraph{Case $b=false$.}
This terminates the loop, thus $\sigma = \sigma'$.
In this case we have that $\sigma \vDash b = false$, by the assumption.
From the wlp rule for while expressions, we get that $\sigma \vDash I$ and $\sigma \vDash ((b = false \land I) \Rightarrow Q) [ w_{i} \mapsto x_{i}]$.
By the preliminary notion, we know that we can instatiate each $w_{i}$ to the variables of $\sigma$,
and then we get $\sigma \vDash (b = false \land I) \Rightarrow Q$.
Now, as $\sigma \vDash I$ and $\sigma \vDash b = false$ we get that $\sigma \vDash Q $, which was exactly what we wanted to show.

\subparagraph{Case $b = true $.}
In this case the loop does not terminate, but executes another iteration of the body.
That means $\sigma \neq \sigma'$, and that $\judges{s',\sigma}{\sigma''}$ and than $\judges{s, \sigma''}{\sigma'}$.

% the execution must first execute $\sigma, s \rightsquigarrow \sigma, b$, as the state does not change at first.
% Then the execution must execute the sequence $b;s$ in $\sigma$, and end in some state $\sigma''$.
% Lastly, from that state executing $s$ again will get us to the final state $\sigma$.

We can prove that $\{WLP(s, Q)\}s\{Q\}$ by proving that $\sigma'' \vDash I$ and $\sigma'' \vDash WLP(s, Q)$,
from where we can use induction on the length of the execution to say that $\sigma' \vDash Q$.

By the assumption we know that $\sigma \vDash b = true$,
and by the wlp rule we have that $\sigma \vDash I$ and
$\sigma \vDash (((b = true \land I) \Rightarrow WLP(s', I)) \wedge ((b = false \wedge I) \Rightarrow Q)) [ w_{i} \mapsto x_{i}]$.
The second part of the conjunction is trivially true, hence we dont consider it in the rest of the proof.
Again we instatiate all variables to those in $\sigma$, and get $\sigma \vDash (b = true \land I) \Rightarrow WLP(s', I))$.
Because we have that $\sigma \vDash b = true$ and $\sigma \vDash I$, we get that $\sigma \vDash WLP(s', I)$.
\\~\\

By IH on the structure we have $\{WLP(s', I)\}s'\{I\}$, hence $\sigma'' \vDash I$.

As the only difference between $\sigma''$ and $\sigma$ is the values of the variables $w_{i}$ then by the preliminary remark we get that $\sigma'' \vDash ((b = true \land I) \Rightarrow WLP(s', I)) [ w_{i} \mapsto x_{i}]$,
which means that we now have all we need to know that $\sigma'' \vDash WLP(s, Q)$.
We now use the induction on the length of the execution to get $\sigma' \vDash Q$,
which is exactly what we wanted to show.
\\~\\
As a consequence of soundness $\vDash P \Rightarrow WLP(s, Q)$ suffices to show partial correctness.
Hence For a VC generator for partial correctness, only an implementation of $WLP$ is needed.

\subsection{Tools}

\subsubsection{SMT solvers and ATP}
\begin{itemize}
  \item Brief overview of what SMT solver and ATP are
  \item Listing the different tools we specifically have used.
  \item expressiveness and effectiveness
\end{itemize}

\subsubsection{Why3}\label{sec:why3}
\begin{itemize}
  \item Brief overview of syntax and usecases
  \item expressive power (coupled with backend)
\end{itemize}

\subsubsection{Isabelle/HOL}\label{sec:isabelle}
\begin{itemize}
  \item Brief overview of syntax and usecases
  \item expressive power (coupled with backend)
\end{itemize}
