As object language we consider a simple imperative language often in the literature referred to as either
IMP or WHILE and from hereon out we will reference our definition as the latter. Notice the language might differ slightly from other definitions but any such derivations should be insignificant.
We consider a language with two basic types, integers and booleans.
\autoref{fig:grammarexpr} shows the grammar for arithmetic and boolean expressions.
An arithmetic expression might either be a variable, an integer, or a suvtraction of two arithmetic expressions.
Notice here than we only define a single binary operation for arithmetic expressions, instead of defining both subtraction and addition.
We do so for a simpler compiler, without loss of expressiveness, as we can define addition in form of subtraction by
$a_{0} + a_{1} = a_{0} - (0 - a_{1})$.

In a similar manner we can define boolean expressions as either $\top$ (true), $\bot$ (false), a negation, a conjunction or a relation. Again we limit relational operators to only a single operator as we can describe the rest in terms of $\leq$ paired with $\neg$.

\colorbox{BurntOrange}{TODO???? We might make a table of the syntactic sugared versions. But is it kind of trivial?}

Likewise with the functionally complete set $\{\wedge, \neg\}$ we can define all possible boolean expressions.

\begin{figure}[h!]
\centering
\begin{grammar}
<aexpr> ::= <identifier>
\alt <integer>
\alt <aexpr> <aop> <aexpr>
\alt '(' <aexpr> ')'

<aop> ::= '$-$'

<bexpr> ::= 'true' | 'false' | \alt '$\neg$'<bexpr>
\alt <bexpr> '$\wedge$' <bexpr>
\alt <aexpr> <rop> <aexpr>
\alt '(' <bexpr> ')'

<rop> ::= '$\le$'
\end{grammar}
\label{fig:grammarexpr}
\caption{Grammar for arithmetic \& boolean expressions}
\end{figure}

Next we want to define assertions in our language, corresponding to a first order logic.
Assertions will be used for two things: 1) the Hoare logic and Verification condition generation, and 2) as regular assertions, which will terminate program execution if the assertion is not valid.
We have a functionally complete set in $\{\forall, \wedge, \neg\}$.

Note here that in other definitions of assertion languages in relation to hoare logic it is common to include ghost variables, which are not allowed in the execution of programs and hence are only allowed in the logic language. In our current state, we do not include such variables.
The grammar for assertions is depicted in \autoref{fig:grammarasserts}.
\begin{figure}[h!]
\centering
\begin{grammar}
<assertion> ::= '$\forall$' <identifier> '.' <assertion>
\alt '$\neg$' <assertion>
\alt <assertion> '$\wedge$' <assertion>
\alt <bexpr>
\end{grammar}
\label{fig:grammarasserts}
\caption{Grammar of assertion language}
\end{figure}

Lastly we define the grammar of statements. A WHILE program is a possibly empty sequence of statements, and a statement can be either a variable binding, an if-else statement, a while loop, an assertion or a skip command. \autoref{fig:grammarstmts} shows the grammar of statements.

\begin{figure}[h!]
\centering
\begin{grammar}
<statements> ::= <statement> ';' <statements> | $\epsilon$

<statement> ::= <identifier> ':=' <aexpr>
\alt 'if' <bexpr> '\{' <statements> '\}' 'else' '\{' <statements> '\}'
\alt 'while' <bexpr> 'invariant' <aexpr> '\{' <statements> '\}'
\alt '\#\{' <assertion> '\}'
\alt 'skip'
\end{grammar}
\label{fig:grammarstmts}
\caption{Grammar for a WHILE program}
\end{figure}

In future work it could be interesting to extend the language with additional types, but for now we keep the structure of the pbject language simple.
