Now we examine the semantics of assertions, in the form of formulas.
As formulas can contain quantifiers, we cannot define the semantics of formulas by a set
of inference rules in terms of operational semantics.
Instead we define it as a set of satisfaction relations $\sigma \vDash f$, which can be seen in \autoref{fig:formulasemantics}.
Note that in the figure $\top$ and $\bot$ denotes that a formula holds or does not hold
respectively, thus it is not used as an indication of normal versus abnormal behaviour
in this case.
It is evident from the rules that \mathbf{true} will hold for all $\sigma: \Sigma$,
whereas \mathbf{false} will never hold.
%This approach will be presented in \ref{sec:iformulas}.

\begin{figure}[h!]
  \begin{align*}
    % terms
    \sigma \vDash \mathbf{true} &\Longleftrightarrow \top \\
    \sigma \vDash \mathbf{false} &\Longleftrightarrow \bot \\
    % and
    \sigma \vDash f_{0} \land f_{1} &\Longleftrightarrow
           (\sigma \vDash f_{0}) \land (\sigma \vDash f_{1}) \\
    % not
    \sigma \vDash \neg f &\Longleftrightarrow \neg (\sigma \vDash f) \\
    % imp
    \sigma \vDash f_{0} \Rightarrow f_{1} &\Longleftrightarrow
           (\sigma \vDash f_{0}) \Rightarrow (\sigma \vDash f_{1}) \\
    % forall
    \sigma \vDash \forall x.f &\Longleftrightarrow
                                \forall n:\mathbb{Z}. (\sigma [x \mapsto n] \vDash f)
  \end{align*}
\caption{Satisfaction relation for evaluation of formulas.}
\label{fig:formulasemantics}
\end{figure}

\textbf{TODO: more about this? or is this fine?}
% \textbf{TODO: I am confused about this last part????}
% see chapter3 of SATnotes.

% \textbf{TODO: remember to write about unbound variables in assertions. (Variables connato be unbound).}

Another interesting notion about evaluation of formulas is the case of unbound variables.
One approach would be to treat abnormal behaviour, such as unbpund variables, as \mathbf{false},
as a formula like $a \leq 0$ would not be valid if $a$ was unbound.
However, in the case of a formula like $\neg (a \leq 0)$ the inner expression would evaluate to
false, following the logic we just presented, and then the entire formula would evaluate to
true. This would indeed be a mistake, thus we need to ensure that the formula is closed before
starting to evaluate it. Therefore evaluation of a formula assumes that the formula is closed.
We will see a solution for implementing this in Why3 in \ref{sec:iformulas}.
