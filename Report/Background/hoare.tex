Floyd–Hoare logic (from now simply refered to as Hoare logic) is a formal system proposed by Tony Hoare in 1969,
based on earlier work from Robert W, which is used to prove correctness of a program.
The core of this formal system is a Hoare triple denoted as $\hoarePQ{s}$
where $P$ and $Q$ are assertions. This notion means that if the precondition $P$ holds in the initial state
and $s$ terminates then $Q$ holds in the halting state of $s$. More formally we write:

$$\hoarePQ{s} = \forall \sigma, \sigma' \in \Sigma. (\sigma \vDash P) \wedge ( \judge{s, \sigma}{\sigma'} \rightarrow (\sigma' \vDash Q))$$

Thus if a hoare triple holds, it is correct wrt to the assertions.

Hoare logic provides axioms and inference rules for an imperative language.
The statements included in these rules corresponds to statements provided in \autoref{sec:syntax}-\autoref{sec:semantics}. \autoref{fig:hoare} shows the inference rules.
Notice that we show the formalization in the traditional sense and do not consider behaviour of unbound variables.

\begin{figure}[h!]

\inference[HSkip]{}
{\hoarePQ{\mathbf{skip}}}

\inference[HAssign]{}
{\hoare{Q[x \mapsto a]}{x := a}{Q}}

\inference[HIf]{\hoare{P \wedge b}{s_0}{Q} & \hoare{P \wedge \neg b}{s_1}{Q}}
{\hoarePQ{ \mathbf{if} \; b \; \{ \; s_0 \; \} \mathbf{else} \; \{ \; s_1 \; \} }}

\inference[HWhile]{\hoare{P \wedge b}{s}{P}}
{\hoare{P}{ \mathbf{while} \; b \; \mathbf{invariant} \; I \; \{ \; s \; \} }{P \wedge \neg b}}

\inference[HSeq]{\hoare{P}{s_0}{R} & \hoare{R}{s_1}{Q}}
{\hoarePQ{s_0 \; ; \; s_1}}

\inference[HCons]{\vDash P \rightarrow P' \hoare{P}{s_0}{Q} & \vDash Q' \rightarrow Q}
{\hoarePQ{s_0}}

\label{fig:hoare}
\caption{Hoare logic inference rules.}
\end{figure}

\textbf{TODO: update to include invariant in while statement?}

The rules are rather intuitive.
For Hskip, the pre and postcondition must be the same.
For Hassign all occurences of x in postcondition Q must be substituted with a.
For conditionals HIf we have that if the precondition P and the conditional check b holds then Q must hold after the then branch $s_{0}$, and if P and not b holds then Q must hold after the else branch $s_1$.
For while the assertion, P is called the invariant and must hold before each iteration of the loop and after the loop. In a similar fashion if the conditional b holds before an iteration of s then the invariant must still hold. And when the loop ends, the condition must not hold any longer.
Sequencing is rather trivial, if P holds before s_0 then R must hold after s_1 Q must hold.
Lastly the conseqeuce rule HCons is used to strengthen a precondition or weaken a postcondtion.
From these rules many additional statements can be derived.
\\~\\
Using these inference rules one can show a program to be partially correct.
The reason we can only show partial correctness is because the transitional sequence of statements might be infinite, i.e. a while loop might not terminate.

MAYBE WRITE ABOUT FINITE TRANSITIVE CLOSURE

To prove total correctness additional information for while-loops have to be included in the logic.
To prove termination a variant $v$ must be provided and it must be subject to a well-founded relation $\prec$.
\[
\inference[HWhile]{\hoare{P \wedge b \wedge v = \xi }{s}{P \wedge \xi}}
{\hoare{P}{ \mathbf{while} \; b \; \mathbf{invariant} \; P \; \mathbf{variant} v, \prec \; \{ \; s \; \} }{P \wedge \neg b}}[wf(\prec)]
\]
where, $\xi$ is a fresh variable. By this total correctness can be achieved.
\\~\\
Hoare logic can be shown to be sound, that is if a Hoare triple is proveable from the inference rules then the hoare triple is valid, i.e. $\vdash \hoarePQ{s}$ implies $\vDash \hoarePQ{s}$. This is an important
property, since we will be unable to derive partial correctness proofs that does not hold. The proof is done on the induction on the derivation of $\vdash \hoarePQ{s}$. We omit this proof, because we dont use Hoare logic in our formalization directly but rather as a consequence for our Verification condition generation.
\\~\\
On the other hand Hoare logic was in 1974\cite{} shown to be relatively complete with respect to the assertion language used. That is hoare logic is no more incomplete than the assertion language. This means that $\vdash \hoarePQ{s}$ implies $\vDash \hoarePQ{s}$ and with an expressive enough language.
Notice furthermore that without multiplication Hoare logic would not be complete.

The proof Cook provided used the notion of Weakest liberal precondition (wlp). Wlp was first introduced by Edgar W. Dijkstra and is defined as follows

$$
\forall P'. P' \Longrightarrow P \; \text{iff} \; \hoare{P'}{s}{Q} \; \text{is valid}
$$

That is $P$ is a weakest precodition for P such that $Q$ will hold in the halting state of $s$. We denote
the weakest liberal precodition as $P = wlp(s,Q)$.
