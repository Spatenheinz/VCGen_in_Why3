
Isabelle is a proof assistant, also known as an interactive theorem provers.
It allows for generic implementations of logical formalisms.
The most used version of Isabelle called Isabelle/HOL (from now referenced solely as Isabelle).
The HOL stands for Higher Order Logic and uses higher order logics and builds on an LCF style.
This means that lemmas can only be proven from previously defined functions and lemmas, corresponding to the inference rules of higher order logic.

Despite the fact that Isabelle, like Why3, is developed in ML, the syntax of Isabelle differs a lot and is fairly extensible in terms of user defined operators etc.
The syntax might me too ``cumbersome'' to define at once and we will go through the relevant parts along the way in \ref{}.

\textbf{TODO: What actually to write????}
