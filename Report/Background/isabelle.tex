As mentioned the initial goal we wanted to see if a fully automated verification of a VCgenerator was possible.
However, we have used an interactive theorem prover to prove some of the lemmas, that we were not able to show automatically.
FOr this we used Isabelle.
Isabelle is a proof assistant, also known as an interactive theorem provers.
It allows for generic implementations of logical formalisms.
The most used version of Isabelle called Isabelle/HOL (from now referenced solely as Isabelle).
The HOL stands for Higher Order Logic and uses higher order logics and builds on an LCF style.
This means that lemmas can only be proven from previously defined functions and lemmas, corresponding to the inference rules of higher order logic. More specifically when a lemma has been proved it a constructor of an abstract type thm is made. This ensure that there are typesafety in proofs.
This should sound familiar to how SMT solvers and ATP's work.
Although Isabelle is a proof assistant it has a high degree of automisation.
One of the sellingpoints is the \texttt{sledgehammer},
which like Why3 can discharge a proof obligation to an SMT solvers and ATPs.
In fact by default it discharges to multiple SMT solvers simultainously.
With this feature one can take control of the proof when necessary, and discharge
the otherwise.
We show how we can use this later in the report.
