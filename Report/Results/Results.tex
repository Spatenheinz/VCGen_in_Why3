\section{Results}\label{sec:results}
In this section we go over the results from the project.
We will cover the findings,
what limitations we encountered,
if we succeeded in making a verified Verification Condition Generator in Why3 and if it is even feasible to do so.

\subsection{Overview of Findings from Implementation}
One of the goals we set out to do, was to investigate the expressiveness and effectiveness of different SMT solvers and ATP's in the why3 backend.
The idea was that we should do this through the implementation of the interpreter and the VC generator.
From \ref{sec:implementation} it should be apparent that most of the properties we want to hold cannot be proved, (atleast from our experiments),
hence we have abandoned the idea of making an analysis of which provers are capable of show specific lemmas.
What we can say from the implementation is that whilst we are able to model some of the lemmas other than the straight forward way,
it is not easy to verify programs in Why3.
Although it might be possible to model the lemmas in a way such that the goals we set out to do can be met, especially with better knowledge of transformations in the Why3 IDE.
We still dont have enough experience with Why3 to use these effectively and from the naming of the transformations, and their fairly limited descriptions in the official manual, one might as well just use Coq when transformations are needed.

\subsection{limitiations for automated verification}
In \textit{WP revisited in Why3} they make heavy use of both transformations and Coq to prove their lemmas,
and whilst it is almost 10 years ago, it seems Why3 is still not at a place where this can be achieved with 100 percent of automated proofs. In fact we would argue, that Why3 were more automated in 2012. The reason for this, can be seen in how most of the Coq proofs are done.
They start by adding a tactic to their Coq proofs, as follows:

\begin{lstlisting}
Require Import Why3.
Ltac ae := why3 ``alt-ergo'' timelimit 5
\end{lstlisting}

By this they are able to interface with Why3 from Coq.
So most of the proofs in the article, massages the proofs and then discharges the rest of the program to a Why3 back-end SMT solver, in this case Alt-ergo.
This feature, is alledgedly still available\cite{TODO:LIERS}. However the files one has to use is nowhere to be found in the folders they are supposed to be in.
If this is a relic of the past or badly documented is hard to tell. In any case, we were not able to get this feature to work.
When also considering that we are not able to discharge the proof obligation to Isabelle, we are in a place where we have less tools available than they had in the 2012 article.

\subsection{Has this project been useful?}
Although we have accomlplished the goals we set out to do, we still have gotten a good reason about or formalization of the weakest precondition calculus and how we can formalize the semantics in a programming language such as Why3.
But to say that our product is useful would be a stretch. Without verifying our formalization we could just aswell have written the VC generator in another language.
For what is it worth we have a partially proved implementation in Isabelle.
We began exploring using Isabelle too late in the project to finish an implementation we could extract.
Had we had more time, we would have finished the formalization and done a side by side comparison of the Why3 and Isabelle.
In any case, it seems like it is ``easier'' to formalize a Verification Condition Generator in Isabelle.

\textbf{TODO: soemthing about roadblocks regarding data-structures for the store}
